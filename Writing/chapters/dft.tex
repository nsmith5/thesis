\section{Introduction}
To describe the process of solidification we will be using the phase field crystal theory. Phase field crystal theory is a type of classical density functional theory. Work by Youssof and Ramakrishnan [cite] and Evans[cite] extended the classical density functional theory of inhomogeneous fluids to the phenomena of solidification in pure materials. While the phase field crystal theory was first produced through purely phenomenological means as a variation on the Swift-Hohenburg model[cite], later work showed that the model can be seen as simplification of the Youssof and Ramakrishnan work on the density functional theory of freezing [cite]. 

The connection between phase field crystal theory and density functional theory is important in that it supplies a microscopic motivation of the parameters in the phase field crystal. CDFT is an important ground for motivating any modifications to the PFC theory. To establish both the language of the phase field crystal model as well as its underlying microscopic origins we'll proceed by deriving the classical density functional theory for a pure material. Later we'll see that this generalizes readily to the case of a binary material and how we can construct equations of motion for the density using the so-called Dynamic Density Functional Theory (DDFT). 

The following derivation will follow the work of Hansan [cite theory of simple liquids / or pep espanol].

\subsection{Deriving the Intrinsic Free Energy Functional}

The central object of classical density functional theory is the density field of the system, $\rho(x)$, defined as, 

\begin{equation}
    \rho(x) = \sum_{i=1}^N \delta^{(3)}(x - x_i).
\end{equation}

To construct a statistical theory for the density field we begin by finding the equilibrium probability by maximizing the entropy subject to the macroscopically available information. Should we take the macroscopically available information to be the existance of an average energy, average particle number and average density field, we arrive at a modification of the grand canonical ensemble, 

\begin{equation}
    f(\mathbf{r}^N, \mathbf{p}^N, N) = \frac{1}{\Xi[\psi(r)]}\frac{\exp\left\lbrace -\beta \mathcal{H}(\mathbf{r}^N, \mathbf{p}^N) + \beta \int dr \psi(r) \rho(r)\right\rbrace}{h^{3N} N! }.
\end{equation}

$\beta$, the Lagrange multiplier associated with the constraint of average energy, we immediately associate with the temperature via $\beta = (k_b T)^{-1}$. Interesting, the contraints of average particle number and density are not independent and their seperate Lagrange multipliers, traditionally associated with the chemical potential $\mu$ and external potential $\phi(x)$, can be combined into a single multiplier, $\psi(r)$, called the \textit{intrinsic chemical potential}. 

The partition function, $\Xi$, is a normalization constant and can be seen as a modification to the grand partition function, 

\begin{equation}
    \Xi[\psi(r)] = \sum_{N=0}^\infty \frac{1}{h^{3N}N!}\int d\mathbf{r}^N d\mathbf{p}^N \exp\left\lbrace -\beta \mathcal{H}(\mathbf{r}^N, \mathbf{p}^N) + \beta \int dr \psi(r) \rho(r)\right\rbrace.
\end{equation}

We can also establish an associated thermodynamic potential in the traditional manner which we will call the grand potential for convenience. 

\begin{equation}
    \Omega[\psi(r)] = -k_b T \ln \Xi[\psi(r)]
\end{equation}

The grand potential produces cumulants of the density field under functional differentiation as we might expect. Specifically, the average density field is, 

\begin{equation}
    \f{\d \Omega[\psi]}{\d \psi(r)} = -\langle \rho(r) \rangle \equiv \rho_1(r),
\end{equation}
And the pair correlation function can be expressed as, 
\begin{equation}
    \f{\d^2 \Omega[\psi]}{\d \psi(r) \d \psi(r^\prime)} = - \beta \langle (\rho(r) - \rho_1(r))(\rho(r^\prime) - \rho_1(r^\prime)) \rangle.
\end{equation}
As noted by Espanol et al [cite], the real underpinning of classical density functional theory is in the combination of these two equations. The first, implies that the average density field is a function of only its conjugate field, the intrinsic chemical potential, and the second implies that that relationship is invertible. To see this note that the Jacobian, 

\begin{equation}
    \f{\d \rho_1(r)}{\d \psi(r^\prime)} = \beta \langle (\rho(r) - \rho_1(r))(\rho(r^\prime) - \rho_1(r^\prime)) \rangle, 
\end{equation}
must be positive semi-definite because of the correlation function on the right hand side. Furthermore, we can make a new thermodynamic potential called the \textit{intrinsic free energy functional} by appling a Legendre transform to the grand potential,

\begin{equation}
    \F[\rho_1(x)] = \Omega[\psi[\rho]] + \int dr \rho_1(r) \psi(r).
\end{equation}

It can be shown [cite Hansen appendix] that $\rho_1(x)$ must be the global minimum of the grand potential, which sets the stage for the methodology of classical density functional theory: if we have a defined intrinsic free energy functional, $\F$, we can find the equilibrium density field by solving the asssociated Euler-Lagrange equation, 

\begin{equation}
    \f{\d \Omega[\rho]}{\d \rho(r)} = 0.
\end{equation}

\subsection{Approximating $\F[\rho]$}

To build a theory a freezing using the classical density functional theory we must first construct an appropriate intrinsic free energy functional. We can split the contribution to the free energy functional into two terms: the ideal and excess. 

\begin{equation}
    \F[\rho] = \F_{id}[\rho] + \F_{ex}[\rho]
\end{equation}

The ideal term is generated from the kinetic energy term in the Hamiltonian. Assuming that there are not momentum dependent interactions in the Hamiltonian this contribution can be computed exactly as, 

\begin{equation}
\F_{id}[\rho] = k_bT \int dr \rho(r) \ln\left(\Lambda^3\rho(r)\right) - \rho(r).
\end{equation}
Where, $\Lambda$, is the thermal deBroglie wavelength, 

\begin{equation}
    \Lambda = \sqrt{\f{h^2}{2\pi m k_b T}}.
\end{equation}
The excess term, on the other hand, typically needs to be approximated in some way because it is generated from the interaction potential term in the Hamiltonian which, for most systems, leads to a completely intractible integral. While some pertubation theories exist to approach this problem, including the cluster expansion [Mayer \& Montrell 1941], it is typically more successful to resort to a judicious choice of closure for the Ornstein-Zernike equation. It can be shown that closures of the Ornstein-Zernike equation are equivalent to summing a subset of terms in the cluster expansion to all orders in the perturbation theory. 

Youssof and Ramakrishnan argued that the HNC or hyper-netted chain approximation was just such a choice of closure for the problem of solidification. The HNC closure is known to give accurate solutions in the case of weak long range potentials such as those seen in plasmas. Though the pair potential of a solidifying system may be short-range, in a solid the short-ranged potentials combine to make an effectively long-ranged periodic potential of the lattice and thus the HNC closure is a good choice.

The HNC closure can be seen as a simple Taylor expansion of the excess free energy functional about a uniform reference fluid of density $\rho_0$,

\begin{equation}
    \F_{ex}[\rho] \approx \F_{ex}[\rho_0] + \int dr \left.\f{\d \F_{ex}}{\d \rho(r)}\right\vert_{\rho_0} \Delta \rho(r) + \f{1}{2} \int dr dr^\prime 
    \Delta \rho(r)\left.\f{\d^2 \F_{ex}}{\d \rho(r) \d \rho(r^\prime)}\right\vert_{\rho_0} \Delta \rho(r^\prime) + ... \,\,\,.
\end{equation}

The excess free energy is the generating functional of direct correlation functions, 

\begin{equation}
    \f{\d^n \F_{ex}[\rho]}{\d \rho(r_1) ... \d \rho(r_n)} = -\beta C^{n}(r_1, ..., r_n).
\end{equation}

The first of which, for a uniform fluid, is the excess contribution to the chemical potential which we may express as the total chemical potential less the ideal contribution, 

\begin{equation}
    \left.\f{\d F_{ex}}{\d \rho}\right\vert_{\rho_0} = \mu_0^{ex} = \mu_0 - k_bT\ln\left(\Lambda^3 \rho_0\right).
\end{equation}

Combining ideal and excess contributions and substracting off the constant reference free energy we arrive at an approximation for the deviation in free energy from the reference. 

\begin{equation}
    \Delta \F [\rho] = k_b T \int dr \rho(r) \ln\left(\f{\rho(r)}{\rho_0}\right) -(1-\beta\mu_0) \Delta \rho(r) - \f{1}{2}\Delta \rho(r) \int dr^\prime C^{(2)}(r, r^\prime) \Delta \rho(r^\prime)
\end{equation}

