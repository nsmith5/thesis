In a many body system in which particles interactions are independent of their velocities, we may split contributions to the free energy into two parts: the ideal and the excess.

\subsubsection{The ideal component of the free energy}
The ideal component comes from the kinetic energy term in the Hamiltonian and is known exactly.

\begin{equation}
    \beta\F_{id}[\rho] = \int dr \rho(r) \l\lbrace\ln\right(\Lambda^3\rho(r)\left) - 1\r\rbrace
\end{equation}
Where,
\begin{description}[labelindent=10pt, labelsep=10pt]
\item[$\Lambda^3$] is the thermal DeBroglie volume
\item[$\rho(r)$] is the number density
\item[$\beta$] is the inverse temperature ($1/k_bT$)
\end{description}

\subsubsection{Expansion of the excess free energy}
The excess component comes from the interaction term in the Hamiltonian and is often not known exactly and must be modelled.
A common technique for modelling the excess free energy is to expand it around uniform fluid reference state.
This reference state is characterized by a number density, $\rho_0$, and chemical potential $\mu_0$.

\begin{equation}
    \beta\F_{ex}[\rho] = \beta\F_{ex}^0 +
    \beta\int dr \left.\frac{\d\F_{ex}}{\d\rho}\right\vert_{\rho = \rho_0} \Delta\rho
    + \beta\f{1}{2}\int dr \int dr^\prime \Delta\rho(r^\prime)\l.\f{\d^2\F_{ex}}{\d\rho(r)\d\rho(r^\prime)}\r\vert_{\rho=\rho_0}\Delta\rho(r^\prime) + ...
\end{equation}

Where,
\begin{description}[labelindent=10pt, labelsep=10pt]
    \item[$\Delta\rho$] is difference from reference density ($\rho(r) - \rho_0$)
    \item[$\F_{ex}^0$] is the free energy of the reference state
\end{description}

The excess free energy is the generating function of direct correlation functions $C^(n)(r_1, ..., r_n)$.
In particular this means that direct correlation functions can be written as,

\begin{equation}
    C^{(n)}(r_1, ..., r_n) = -\beta \f{\d^{(n)}\F_{ex}[\rho]}{\d\rho(r_1)...\d\rho(r_n)},
\end{equation}
and our previous expansion can be rewritten in terms of these direct correlation functions.

\begin{equation}
    \beta\F_{ex}[\rho] = \beta\F_{ex}^0 - \int dr C^{(1)}_0(r) \Delta\rho
    + \beta\f{1}{2}\int dr \int dr^\prime \Delta\rho(r^\prime)C^{(2)}_0(r, r^\prime)\Delta\rho(r^\prime) + ...
\end{equation}

In the absence of an external field the single particle direct correlation function of the reference system is simply $\beta\mu^{ex}_0$.

\subsubsection{Total Free Energy}
If we now add in the ideal contribution to the free energy and take advantage of the fact that the excess chemical potential of the reference fluid can be written as the total chemical potential minus the ideal contribution,

\begin{equation}
    \mu^{ex}_0 = \mu_0 - \mu^{id}_0 = \mu_0 - k_bT\ln(\Lambda^3\rho_0).
\end{equation}
to find an expression for the total free energy:

\begin{equation}
    \beta\F[\rho] = \beta\F_0 + \int\,dr \l\lbrace \Delta\rho \ln\l(\f{\Delta\rho}{\rho_0}\r) - (1 - \mu_0)\Delta\rho \r\rbrace
    - \f{1}{2}\int\,dr\int\,dr^\prime \Delta\rho(r) C^{(2)}_0(r, r^\prime) \Delta\rho(r^\prime)
\end{equation}

\subsubsection{Smooth atom approximation and the PFC Free Energy}

To construct the phase-field crystal free energy we assume that the density fluctations, $\Delta\rho(r)$, are small and expand the logarithm term to quartic order in the fluctuations.
Furthermore, we nondimensionalize the free energy by scaling out the reference density, $\rho_0$, and changing variables to $n(r) = \Delta\rho(r)/\rho_0$.

\begin{equation}
    \f{\beta\F[n]}{\rho_0} = \f{\beta\F_0}{\rho_0} +
    \int dr \mu_0 \rho_0 n(r) + \f{n^2(r)}{2} - \f{n^3(r)}{6} + \f{n^4(r)}{12}
    -\f{\rho_0}{2} \int dr \int dr^\prime n(r) C^{(2)}_0(r, r^\prime) n(r^\prime)
\end{equation}

At this point we note that the constant and linear terms in the free energy can be removed without changing the properties of the functional and we are left with the following minimal free energy functional:

\begin{equation}
    \f{\beta\F}{\rho_0} = \int dr \f{n^2(r)}{2} - \f{n^3(r)}{6} + \f{n^4(r)}{12}
    -\f{1}{2} \int dr \int dr^\prime n(r) C^{(2)}(r, r^\prime) n(r^\prime).
\end{equation}

Note, that by convention a factor of the reference density is absorbed into the pair correlation function in the PFC free energy functional.
