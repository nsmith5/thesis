\subsection{Introduction}
% Why are we using classical DFT?
% what do we care about?
% blah blah blah

\subsection{Deriving the Intrinsic Free Energy Functional}

To derive the intrinsic free energy functional we start with the grand partition function for a single component system in the semi-classical limit.

\begin{equation}
    \Xi = \sum_{N=0}^\infty \f{e^{N\beta\mu}}{h^{3N} N!} 
                   \int d\mathbf{r}^Nd\mathbf{p}^N e^{ -\beta \mathcal{H}}  
\end{equation}

Where the Hamiltonian, $\mathcal{H}$, has contributions from kinetic energy, interaction potential and an external potential.

\begin{equation}
    \mathcal{H} = \sum_{i=1}^N \f{p_i^2}{2m} + V(r_1, r_2, ..., r_N) + \sum_{i=1}^N \phi(r_i)
\end{equation}

Note that we explicitly assume that the interactions only depend on the particle positions and not their momenta, this allows us to complete the momentum integral in the partition function. 

\begin{equation}
    \Xi = \sum_{N=0}^\infty \f{1}{\Lambda^{3N}N!} 
                   \int d\mathbf{r}^N e^{ -\beta \left(V(r_1, r_2, ...,r_N) - \sum_{i=1}^N(\mu - \phi(r_i))\right)} 
\end{equation}

Note now, that we're trying to construct a theory for the density field of our material, $\rho(r)$. The density field is defined as a sum of Dirac delta functions at the positions of the particles. That is, we would like to write the grand partition function in terms the density field and use the grand partition function as a generating functional for the mean, fluctuations and other moments of the density field.

\begin{equation}
    \rho(r) \equiv \sum_{i=1}^N \delta^{(3)}(r_i - r)
\end{equation}

We start along this path by noting that the external potential term can be rewritten in terms of the density field as follows, 

\begin{equation}
    \sum_{i=1}^N (\mu - \phi(r_i)) = \int dr (\mu - \phi(r))\rho(r)
\end{equation}

In this form, we see that the combination $\mu - \phi(r)$ forms a field that is conjugate to the density field that we will call the \textit{intrinsic chemical potential}, $\psi(r)$. What we mean by conjugate is that repeated functional derivative of the grand partition function will generate correlation functions of the density field. 

\begin{equation}
\f{1}{\beta^N \Xi[\psi(r)]}\f{\d^n \Xi[\psi(r)]}{\d \psi(r_1) ... \d \psi(r_n)} = \left\langle \rho(r_1) ...\rho(r_n) \right\rangle
\end{equation}

It is important to note that the partition function has two distinct contibutions: one from the ideal contibutions to the free energy and one from the interaction term which is called the \textit{excess term}. 

\begin{equation}
    \Xi[\psi(r)] = 
\end{equation}