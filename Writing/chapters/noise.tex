When using Langevin equations to study non-equilibrium statistical mechanics the noise strength can be linked to the transport coefficients through a generalization of the Einstein relation, $D = \mu k_bT$. The typical strategy for deriving such a relationship is to evaluate the equilibrium pair correlation function by two separate methods: the equilibrium partition functional and the equation of motion.

While the equilibrium partition functional gives pair correlation through the typical statistical mechanical calculation, the equation of motion can be used to derive a dynamic pair correlation function that must be equal to the equilibrium pair correlation function in the long time limit.

In what follows we'll look at how to formulate a generalized Einstein relation from a generic Langevin equation and then calculate two specific examples using Model A dynamics and a $phi^4$ theory and Time Dependent Density Functional Theory (TDDFT) and a general Helmholtz free energy.

\section{Generalized Einstein Relations in an Arbitrary Model}

We start by considering a set of microscopic observables, $a_i(r, t)$, that are governed by a nonlinear Langevin equation,

\begin{equation}
	\f{\partial \mathbf{a}(r, t)}{\partial t} = F[\mathbf{a}(r,t)] + \boldsymbol{\xi}(r,t).
\end{equation}

Where, $\mathbf{a}$, denotes a vector of our fields of interest. These microscopic equation of motion may have been derived from linear response, projection operators or some other non-equilibrium formalism. We assume that the random driving force, $\boldsymbol{\xi}(r, t)$ is unbaised, Gaussian noise that is uncorrelated in time.

\begin{gather}
	\l\langle \boldsymbol{\xi}(r,t)\r\rangle = 0 \\
	\l\langle \boldsymbol{\xi}(r,t)\boldsymbol{\xi}^\dagger(r^\prime,t^\prime)\r\rangle =
	\mathbf{L}(r, r^\prime)\d(t-t^\prime)
\end{gather}

We wish to constrain the form of the covariance matrix, $\mathbf{L}$ of $\boldsymbol{\xi}$ by demanding that the solution to the Langevin equation eventually decays to equilibrium and that correlations in equilibrium are given by Boltzmann statistics.

We begin by linearizing the equation of motion about an equilibrium solution, $\mathbf{a}(r, t) = \mathbf{a}_{eq}(r) + \hat{\mathbf{a}}(r, t)$.

\begin{equation}
	\f{\partial \hat{\mathbf{a}} (r, t)}{\partial t} = \mathbf{M}(r, r^\prime) \ast \hat{\mathbf{a}}(r^\prime, t) + \boldsymbol{\xi}(r, t)
\end{equation}

Where, $\ast$ denotes an inner product and integration over the repeated variable. eg:

\begin{equation}
	\mathbf{M}(r, r^\prime)\ast \hat{\mathbf{a}}(r^\prime) = \sum_j \int\,dr^\prime M_{ij}(r, r^\prime) \hat{a}_j(r^\prime).
\end{equation}

We can formally solve our linearized equation of motion.

\begin{equation}
	\hat{\mathbf{a}}(r, t) = e^{\mathbf{M}(r, r^\prime)t}\ast\hat{\mathbf{a}}(r^\prime, 0) + \int_0^t d\tau\, e^{\mathbf{M}(r, r^\prime)(t-\tau)} \ast \boldsymbol{\xi} (r^\prime, \tau)
\end{equation}

We can use this formal solution to evaluate the equilibrium pair correlation function.

\begin{align}
	\l\langle \hat{\mathbf{a}}(r, t)\hat{\mathbf{a}}^\dagger(r^\prime, t^\prime) \r\rangle &= e^{\mathbf{M}(r, r_1)t}\ast\l\langle \hat{\mathbf{a}}(r_1, 0)\hat{\mathbf{a}}^\dagger(r_2, 0) \r\rangle \ast e^{\mathbf{M}^\dagger(r^\prime, r_2)t^\prime} \nonumber \\
	&+ \int_0^t \int_0^{t^\prime}d\tau d\tau^\prime\, e^{\mathbf{M}(r, r_1)(t-\tau)}\ast\l\langle \boldsymbol{\xi}(r_1, 0)\boldsymbol{\xi}^\dagger(r_2, 0) \r\rangle \ast e^{\mathbf{M}^\dagger(r^\prime, r_2)(t^\prime-\tau^\prime)}
\end{align}

It is important to note that every eigenvalue of $\mathbf{M}$ must be negative for our solution to decay to equilibrium in the long time limit (eg. $lim_{t\rightarrow\infty}\hat{\mathbf{a}}(r, t) = 0$) and as such the first term in our dynamic correlation function won't contribute to the equilibrium pair correlation. This is as we might expect as the first term holds the contributions to the correlation function from the initial conditions. The second term can be evalutated by substituting the noise correlation and evaluating the delta function.

\begin{equation}
	\mathbf{\Gamma}(r, r^\prime) = \lim_{t\rightarrow\infty} \l\langle \hat{\mathbf{a}}(r, t)\hat{\mathbf{a}}^\dagger(r^\prime, t) \r\rangle =
	\int_0^\infty dz \,e^{\mathbf{M}(r, r_1)z}\ast\mathbf{L}(r_1, r_2)\ast e^{\mathbf{M}^\dagger(r^\prime, r_2)z}
\end{equation}

Considering the product $\mathbf{M}(r, r_1)\ast\mathbf{\Gamma}(r_1, r^\prime)$ and performing an integration by parts gives the final generalized Einstein relation.

\begin{equation}
	\mathbf{M}(r, r_1)\ast\mathbf{\Gamma}(r_1, r^\prime) + \mathbf{\Gamma}(r, r_1)\ast\mathbf{M}^\dagger(r_1, r^\prime) = -\mathbf{L}(r, r^\prime)
\end{equation}

\section{Fluctuation Dissipation in Model A}

As a first example of calculating an Einstein relation consider the following free energy functional under non-conservative, dissipative dynamics.

\begin{gather}
\beta \F[\phi] = \int dr \left\lbrace \f{1}{2}\vert \nabla \phi(x) \vert^2 + \f{r}{2}\phi^2(x) + \f{u}{4!}\phi^4(x)  + h(x)\phi(x)\right\rbrace \\
\f{\partial \phi(x,t)}{\partial t} = -\Gamma \left(\f{\delta \beta \F[\phi]}{\delta \phi(x)}\right) + \xi(x, t)
\end{gather}

The random driving force, $\xi$, is Gaussian white noise with some scalar noise strength $\lambda$.

\begin{align}
\l\langle \xi (x, t) \r\rangle &= 0 \\
\l\langle \xi (x, t) \xi(x^\prime, t^\prime) \r\rangle  &= \lambda \delta(x - x^\prime) \delta (t - t^\prime)
\end{align}

The fluctuation-dissipation theorem will ultimately show us that the noise strength $\lambda$ and the transport coefficient $\Gamma$ are related when the system is close to an equilibrium state by $\lambda = 2\Gamma$. To see how this might come about we start by evaluating the equilibrium pair correlation function using the partition functional of our theory

\subsection{The partition function route}

In equilibrium the probability of particular field configuration is given by the Boltzmann distribution.

\begin{equation}
\mathcal{P}_{eq}[\phi] = \f{e^{-\beta\F[\phi]}}{\mathcal{Z}[h(x)]}
\end{equation}

Where, $\mathcal{Z}[h(x)]$ is the partition functional and is given by a path integral over all field configurations.

\begin{equation}
\mathcal{Z}[h(x)] = \int \mathcal{D}[\phi] e^{-\beta\F[\phi]}
\end{equation}

Evaluation of the partition function is of some importance because it plays the role of a moment generating function.

\begin{equation}\label{gen}
\f{1}{\Z[h]}\f{\delta^n \Z[h]}{\delta h(x_1)...\delta h(x_n)} = \langle \phi(x_1)...\phi(x_n)\rangle
\end{equation}

In general the partition function cannot be computed directly, but in the special case of Gaussian free energies it can. To that end we consider expanding phi around an equilibrium solution, $\phi(x) = \phi_0 + \Delta\phi(x)$, and keeping terms to quadratic order in the free energy.

\begin{equation}
\beta\F[\Delta\phi] = \int dr \,\left\lbrace \f{1}{2}\Delta\phi(x) \left(r - \nabla^2 + \f{u}{2}\phi_0^2\right) \Delta\phi(x) - h(x)\Delta\phi(x) \right\rbrace
\end{equation}

Here the partition function is written in a suggestive form. As stated previously, functional integrals are difficult to compute in general, but Gaussian functional integrals do have a solution.

\subsubsection{Gaussian Functional Integrals}

Consider a functional integral of the following form.

\begin{equation}
\Z[h(x)] = \int \D[\phi] \exp\left\lbrace - \int dx \int dx^\prime \left[ \f{1}{2}\phi(x) \mathbf{K}(x, x^\prime) \phi(x^\prime)\right] +  \int dx \left[h(x) \phi(x)\right]\right\rbrace
\end{equation}

This integral is simply the continuum limit of a multivariable Gaussian integral,

\begin{equation}
\Z[\mathbf{h}] = \int \prod_i dx_i \exp \left\lbrace - \f{1}{2}\sum_i \sum_j x_i\, \mathbf{K}_{ij}\, x_j  + \sum_i h_i x_i\right\rbrace,
\end{equation}
For which the solution is,

\begin{equation}
\Z[\mathbf{h}] = \sqrt{\f{2\pi}{\det(\mathbf{K})}} \exp\left\lbrace \f{1}{2} \sum_i \sum_j h_i \mathbf{K}_{ij}^{-1} h_j\right\rbrace.
\end{equation}
In the continuum limit, the solution has an analogous form.

\begin{equation}\label{part}
\Z[h(x)] \propto \exp\left\lbrace \int dx \int dx^\prime \left[ \f{1}{2}h(x) \mathbf{K}^{-1}(x, x^\prime) h(x^\prime)\right] \right\rbrace
\end{equation}
Where $\mathbf{K}^{-1}$ is defined by,

\begin{equation}
\int dx^\prime \mathbf{K}(x, x^\prime)\mathbf{K}^{-1}(x^\prime, x^{\prime\prime}) = \delta(x - x^{\prime\prime}).
\end{equation}
Ultimately, we don't need to worry about the constant of proportionality in equation \ref{part} because we'll be dividing this contribution when calculating correlation functions.

\subsubsection{Computing the Pair correlation function in the Gaussian approximation}

To compute the pair correlation function we use the Fourier space variant of the partition function,

\begin{equation}
\Z[\tilde{h}(k)] \propto \exp\left\lbrace \f{1}{2}\int dk\,\f{h(k)h^{*}(k)}{r + \f{u}{2}\phi_0^2 +  \vert k \vert^2}\right\rbrace.
\end{equation}
The pair correlation function, $\langle \Delta\tilde{\phi}(k)\Delta\tilde{\phi}^{*}(k)\rangle$, is then computed using equation \ref{gen}.

\begin{equation}
\l\langle \Delta\fphi(k)\Delta\fphi^{*}(k^\prime) \r\rangle = \f{2\pi \delta(k+k^\prime)}{r + \f{u}{2}\phi_0^2 + \vert k \vert^2}
\end{equation}

\subsection{The Equation of Motion Route}

The equation of motion supplies a second method for evaluating the pair correlation function in equilibrium.

\begin{equation}
\f{\partial \phi}{\partial t} = -\Gamma\left((r-\nabla^2)\phi(x,t) + \f{u}{3!}\phi^3(x,t)\right) + \xi(x, t),
\end{equation}

Our equation of motion, can be linearized around an equilibrium solution, $\phi_0$, just as we did in the partition function route to the pair correlation function. In a similar vain, we will Fourier transform the equation of motion as well.

\begin{equation}
\f{\partial \Delta\fphi(k, t)}{\partial t} = -\Gamma\left((r + \f{u}{2}\phi_0 + \vert k \vert^2)\Delta\fphi(k,t)\right) + \xi(x,t)
\end{equation}
This equation can be solved formally using a Green's function solution.

\begin{equation}
\Delta\fphi(k, t) = e^{-\Omega t}\Delta\fphi(k, 0) + e^{-\Omega t}\int_0^t d\tau \,e^{\Omega \tau} \fxi(k, \tau)
\end{equation}
Where, $\Omega = \Gamma (r + \f{u}{2}\phi_0^2 + \vert k \vert^2)$.

\subsubsection{Computing the Pair correlation function}

We now have the tools to compute the dynamical pair correlation function, $\langle \Delta\fphi(k, t)\Delta\fphi(k^\prime, t^\prime) \rangle $, and in limit as time goes to infinity, we will have another expression for the equilibrium pair correlation function. We begin by inserting the Green's function solutions into the pair correlation expression

\begin{align}
\l\langle \Delta\fphi(k, t) \Delta\fphi(k^\prime, t^\prime)\r\rangle &=  e^{-\Omega(t+t^\prime)}\Delta\fphi(k, 0)\Delta\fphi(k^\prime, 0) \nonumber \\
 &+ e^{-\Omega (t+t^\prime)} \int_0^t d\tau \int_0^{t^\prime} d\tau^\prime e^{\Omega(\tau+\tau^\prime)}\l\langle \fxi(k, \tau) \fxi(k^\prime, \tau^\prime) \r\rangle.
\end{align}
Using the noise correlation we can compute the second term to find the final form of the dynamic correlation function.

\begin{align}
	\left\langle \Delta\fphi(k, t) \Delta\fphi(k^\prime, t^\prime)\right\rangle &=  e^{-\Omega(t+t^\prime)}\left(\Delta\fphi(k, 0)\Delta\fphi(k^\prime, 0) - \f{2\pi\delta(k+k^\prime)\lambda}{2\Omega}\right) \nonumber \\
	&+ \f{2\pi\delta(k+k^\prime)\lambda}{2\Omega}e^{-\Omega \vert t-t^\prime \vert}
\end{align}

Setting, $t = t^\prime$ and taking the limit as $t\rightarrow\infty$ we recover another form for the equilibrium pair correlation function.

\begin{equation}
	\left\langle \Delta\fphi(k, t) \Delta\fphi(k^\prime, t^\prime)\right\rangle = \f{2\pi\delta(k+k^\prime)\lambda}{2\Gamma(r + \f{u}{2}\phi_0^2 + \vert k\vert^2)}
\end{equation}

\subsection{Remarks}

Comparing with the result we got from the partition function route and the equation of motion route we see that for our answers to be equal we must have $\lambda = 2\Gamma$. It should be noted that this answer may seem to differ from other definitions of the fluctuation-dissipation theorem which state that $\lambda = 2k_b\,T\Gamma$. The discrepancy comes from what we mean by the coefficent $\Gamma$ and how we write the equation of motion. If we write the equation of motion as in equation \ref{eom}, the coefficient $\Gamma$ is the traditional Onsager transport coefficient and we recover traditional fluctuation-dissipation theorem.

\begin{equation}\label{eom}
	\f{\partial \phi(x,t)}{\partial t} = -\Gamma \left(\f{\delta \F[\phi]}{\delta \phi(x)}\right) - \xi(x,t)
\end{equation}

Comparing with our result we see that the factor of $k_bT$ is absorbed into our definition of the transport coefficient $\Gamma$.

\section{Fluctuation Dissipation Theorem in Dynamic Density Functional Theory}

In dynamic density functional theory (DDFT) we have an equation of motion of the following form,

\begin{equation}
	\f{\partial \rho(r, t)}{\partial t} = D_0 \nabla \cdot \l[\rho(r,t)\nabla \l(\f{\d \F[\rho]}{\d \rho}\r)\r] + \xi(r, t)
\end{equation}

Where, $D_0$ is the equilibrium diffusion constant and $\xi$ is the stochastic driving force. We assume once again that the driving force has no bias, but we now allow the noise strength to be a generic linear operator $\mathcal{L}$.

\begin{align}
	\langle \xi(r,t) \rangle &= 0 \\
	\l\langle \xi(r, t) \xi(r^\prime, t^\prime) \r\rangle &= \mathcal{L} \d (r-r^\prime) \d (t -t^\prime)
\end{align}

We ask ourselves now, what constrains can we apply to this operator if our system must decay to a Boltzmann distribution in equilibrium?

\subsection{Pair Correlation from the Partition Functional}

Just like with the $phi^4$ model we want to expand our free energy functional around an equilibrium solution. In this case our free energy functional is generic so this expansion is purely formal.

\begin{equation}
	\F[\rho] = \F_{eq} + \beta\int dr \l(\l.\f{\d \F[\rho]}{\d \rho(r)}\r)\r\vert_{\rho_{eq}}\Delta\rho(r) + \f{1}{2} \int dr \int dr^\prime \Delta\rho(r) \l(\l.\f{\d^2 \F[\rho]}{\d \rho(r) \d \rho(r^\prime)}\r)\r\vert_{\rho_{eq}} \Delta\rho(r^\prime)
\end{equation}

The first term we can neglect as it adds an overall scale to the partition function that will not affect any of moments. Second moment only shifts the average so we can ignore it as well and so we're left with a simple quadratic free energy once again.

\begin{equation}
	\F[\rho] = \f{1}{2}\int dr \int dr^\prime \Delta \rho(r) H(r, r^\prime) \Delta \rho(r^\prime)
\end{equation}

Where, $H(r, r^\prime)$ is the second functional derivative of the free energy functional in equilibrium. Computing the pair correlation function from the partition function yields, as might be expected,

\begin{equation}
	\l\langle \Delta\rho(r) \Delta\rho(r^\prime) \r\rangle = H(r, r^\prime)
\end{equation}
