\subsubsection{Gaussian Functional Integrals}

In the study of the statistical physics of fields we often encounter functional integrals of the form,

\begin{equation}
\Z[h(x)] = \int \D[\phi] \exp\left\lbrace - \int dx \int dx^\prime \left[ \f{1}{2}\phi(x) \mathbf{K}(x, x^\prime) \phi(x^\prime)\right] +  \int dx \left[h(x) \phi(x)\right]\right\rbrace.
\end{equation}

Solutions to this integral are not only important in there own right but are also the basis perturbative techniques. The detail of how to solve this integral can be found in \cite{Kardar} and are repeated here for the convenience of the reader.

This integral is simply the continuum limit of a multivariable Gaussian integral,

\begin{equation}
\Z[\mathbf{h}] = \int \prod_i dx_i \exp \left\lbrace - \f{1}{2}\sum_i \sum_j x_i\, \mathbf{K}_{ij}\, x_j  + \sum_i h_i x_i\right\rbrace,
\end{equation}
For which the solution is,

\begin{equation}
\Z[\mathbf{h}] = \sqrt{\f{2\pi}{\det(\mathbf{K})}} \exp\left\lbrace \f{1}{2} \sum_i \sum_j h_i \mathbf{K}_{ij}^{-1} h_j\right\rbrace.
\end{equation}
In the continuum limit, the solution has an analogous form.

\begin{equation}\label{part}
\Z[h(x)] \propto \exp\left\lbrace \int dx \int dx^\prime \left[ \f{1}{2}h(x) \mathbf{K}^{-1}(x, x^\prime) h(x^\prime)\right] \right\rbrace
\end{equation}
Where $\mathbf{K}^{-1}$ is defined by,

\begin{equation}
\int dx^\prime \mathbf{K}(x, x^\prime)\mathbf{K}^{-1}(x^\prime, x^{\prime\prime}) = \delta(x - x^{\prime\prime}).
\end{equation}
Ultimately, we don't need to worry about the constant of proportionality in equation \ref{part} because we'll be dividing this contribution when calculating correlation functions.
