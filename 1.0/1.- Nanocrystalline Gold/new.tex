\documentclass[11pt]{article}
\usepackage[margin=1in]{geometry}
\usepackage{amsmath}
\usepackage{array}

\title{Modelling multi-step nucleation with the phase field crystal model}
\author{Nathan Smith}
\date{\today}

\newenvironment{conditions}
	{\par\vspace{\abovedisplayskip}\noindent\begin{tabular}{>{$}l<{$} @{${}={}$} l}}
  	{\end{tabular}\par\vspace{\belowdisplayskip}}

\begin{document}
\maketitle

\section{Introduction}

\begin{itemize}
 \item[-] why should we care about multistep nucleation
 \item[-] why would pfc be a lens into these processes
 \item[-] what are some example of multistep nucleation
\end{itemize} 
 
\section{Free Energy Functional}

To model a binary solution we begin with  


\begin{itemize}
 \item[-] where does the free energy functional we're using come from? 
 \item[-] what simplifications do we make 
 \item[-] how do we build a functional to describe aqueous solutions?
 \item[-] what are the important characteristics of these functionals
\end{itemize}

\section{Equilibrium Properties}

\begin{itemize}
\item Describe the variational method used to make the phase diagram
\item Show a phase diagram
\item Show metastable phase diagram
\item Show the free energy land scape and describe pathways over that landscape to nucleation
\end{itemize}

\section{Dynamics}

\begin{itemize}
\item Show the equations of motion
\item Describe the algorithm needed to simulate those equations of motion
\item Show a really really pretty picture
\item Show a size distribution of nano particles
\item Show energy pathway for a nucleus
\end{itemize}

\end{document}
