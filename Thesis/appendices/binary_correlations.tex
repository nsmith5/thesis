When developing the binary PFC model we often change variables from $\A$ and
$\B$ to $n$ and $c$.  This change of variable is helpful in identifying the
results of the PFC theory with established results in the field as
concentration and total density are more commonly used in the field of material
science. Computing the bulk terms (ie., $\Delta\F_{mix}[n, c]$ and
$\Delta\F_{id}[n]$ from equation \ref{binary_mixing} and \ref{binary_ideal}) is
a matter of substitution and simplification but computing the change of variables for excess free
energy can be more subtle. When computing the pair correlation
terms, careful application of our assumption that $c$ varies over a much longer
length scale than $n$ must be applied to get the correct solution. The goal,
ultimately, is to find $C_{n n}$, $C_{n c}$, $C_{c n}$ and $C_{c c}$ in the
following expression, 

[{\color{ForestGreen} Stopped reviewing here! continue hhere in future}]

%
\begin{gather}\label{1}
      \Delta \A \,\rho_0 C_{AA} \ast \Delta \A 
    + \Delta \A \,\rho_0 C_{AB} \ast \Delta \B 
    + \Delta \B \,\rho_0 C_{BA} \ast \Delta \A 
    + \Delta \B \,\rho_0 C_{BB} \ast \Delta \B = \\ \nonumber
       \l(n \,C_{nn} \ast n 
        + n \,C_{nc} \ast \Delta c 
        + \Delta c \,C_{cn}\ast n 
        + \Delta c \,C_{cc} \ast \Delta c\r).
\end{gather}
%
We begin by rewriting $\Delta \B$,
%
\begin{align*}
  \Delta \B &= \rho c - \rho_0 c_0 \\
        &= \rho c - \rho c_0 + \rho c_0 - \rho_0 c_0 \\
        &= \Delta \rho c + \rho_0 \Delta c,
\end{align*}
%
Followed by rewriting $\Delta \A$,
%
\begin{align*}
  \Delta \A &= \rho (1 - c) - \rho_0 (1 - c_0) \\
        &= \Delta \rho (1 - c) - \rho_0 \Delta c.
\end{align*}
%
With those forms established, we can expand $\Delta \B \,C_{BB} \ast \Delta \B$:
%
\begin{align}\label{this}
  \Delta \B C_{BB} \ast \Delta \B &= \l(\Delta \rho c + \rho_0 \Delta c \r) C_{BB} \ast \l(\Delta \rho c + \rho_0 \Delta c\r) \nonumber\\
                          &= \Delta \rho c \,C_{BB} \ast \l( \Delta \rho c\r) \nonumber\\
                          &+ \rho_0 \Delta c \,C_{BB} \ast \l( \Delta \rho c \r) \\
                          &+ \rho_0 \l(\Delta \rho c\r) \,C_{BB} \ast \Delta c \nonumber\\
                          &+ \rho_0^2 \Delta c \,C_{BB} \ast \Delta c. \nonumber
\end{align}
%
If we examine one term in this expansion in detail, we note that we can
simplify by using the long wavelength approximation for the concentration
field,
%
\begin{align}
  \Delta \rho c \, C_{BB} \ast \Delta \rho c &= \Delta \rho(r) c(r) \int dr^\prime C_{BB}(r - r^\prime) \Delta \rho(r^\prime) c(r^\prime) \nonumber \\
                                     &\approx \Delta \rho(r) c^2(r) \int dr^\prime C_{BB}(r - r^\prime) \Delta \rho(r^\prime).
\end{align}
%
This is because the concentration field can be considered ostensibly constant
over the length scale in which $C_{BB}(r)$ varies. Recall that the pair
correlation function typically decays to zero on the order of several particle
radii. Using this approximation we can rewrite equation \ref{this} as,
%
\begin{align}
  \Delta \B \, C_{BB} \ast \Delta \B &= \Delta \rho \l(c^2 \,C_{BB}\r) \ast \Delta \rho \nonumber\\
                             &+ \rho_0 \Delta c \l(c \,C_{BB}\r) \ast \Delta \rho c \\
                             &+ \rho_0 \Delta \rho \l(c \,C_{BB}\r) \ast \Delta c \nonumber\\
                             &+ \rho_0^2 \Delta c \,C_{BB} \ast \Delta c. \nonumber
\end{align}
%
Repeating this procedure with the remaining three terms and then regrouping we
can easily identify the required pair correlations.\footnote{Note that we may
also take advantage of the fact that $C_{AB} = C_{BA}$.}
%
\begin{gather}
  C_{nn} = \rho_0 \l( c^2 \, C_{BB} + (1 - c)^2 \, C_{AA} + 2c(1-c)\,C_{AB}\r) \\
  C_{nc} = C_{cn} = \rho_0 \l( c\,C_{BB} - (1-c)\,C_{AA} + (1 - 2c) \, C_{AB} \r) \\
  C_{cc} = \rho_0 \l( C_{BB} + C_{AA} - 2 C_{AB} \r)
\end{gather}

