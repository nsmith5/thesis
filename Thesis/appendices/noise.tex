\label{appendix:noise}

When using Langevin equations to study non-equilibrium statistical mechanics,
the noise strength can be linked to the transport coefficients through a
generalization of the Einstein relation. The generalization was first developed
by Onsager and Machlup \cite{OnsagerMachlup}. The typical strategy for deriving
such a relationship is to evaluate the equilibrium pair correlation function by
two separate methods: the equilibrium partition functional and the equation of
motion\footnote{For considerations far from equilibrium see \cite{Lax, Ronis,
Fox_and_Uhlenbeck}}.

While the equilibrium partition functional gives pair correlation through the
typical statistical mechanical calculation, the equation of motion can be used
to derive a dynamic pair correlation function that must be equal to the
equilibrium pair correlation function in the long time limit.

In what follows we'll look at how to formulate a generalized Einstein relation
from a generic Langevin equation and then calculate two specific examples using
Model A dynamics with a $\phi^4$ theory and Time Dependent Density Functional
Theory (TDDFT) with a general Helmholtz free energy.

%%%%%%%%%%%%%%%%%%%%%%%%%%%%%%%%%%%%%%%%%%%%%%%%%%%%%%%%%%%%%%%%
\section{Generalized Einstein Relations in an Arbitrary Model} %
%%%%%%%%%%%%%%%%%%%%%%%%%%%%%%%%%%%%%%%%%%%%%%%%%%%%%%%%%%%%%%%%

We start by considering a set of microscopic observables, $a_i(r, t)$, that are
governed by a nonlinear Langevin equation,
%
\begin{equation} 
    \f{\partial \mathbf{a}(r, t)}{\partial t} = 
        F[\mathbf{a}(r,t)]
      + \boldsymbol{\xi}(r,t).
\end{equation}
%
Where, $\mathbf{a}$, denotes a vector of our fields of interest. These
microscopic equation of motion may have been derived from linear response,
projection operators or some other non-equilibrium formalism. We assume that
the random driving force, $\boldsymbol{\xi}(r, t)$ is unbiased, Gaussian noise
that is uncorrelated in time,
%
\begin{gather}
    \mean{\boldsymbol{\xi}(r,t)} = 0, \\ 
    \label{eq:noise_form} 
    \mean{\boldsymbol{\xi}(r,t) \boldsymbol{\xi}^\dagger(r^\prime,t^\prime)} =
        \mathbf{L}(r, r^\prime)\d(t-t^\prime).
\end{gather}
%
This assumption is justified by positing that the stochastic driving force is
the aggregated affect of many random microscopic processes that satisfy the
central limit theorem so we may assume a Gaussian form. We wish to constrain
the form of the covariance matrix, $\mathbf{L}$, by demanding that the solution
to the Langevin equation eventually decays to equilibrium and that correlations
in equilibrium are given by Boltzmann statistics.

We begin by linearizing the equation of motion about an equilibrium solution,
$\mathbf{a}(r, t) = \mathbf{a}_{eq}(r) + \hat{\mathbf{a}}(r, t)$.
%
\begin{equation}
    \f{\partial \hat{\mathbf{a}} (r, t)}{\partial t} =
        \mathbf{M}(r, r^\prime) \ast \hat{\mathbf{a}}(r^\prime, t) +
\boldsymbol{\xi}(r, t) 
\end{equation}
%
Where, $\ast$ denotes an inner product and integration over the repeated
variable. eg:
%
\begin{equation}
    \mathbf{M}(r, r^\prime)\ast \hat{\mathbf{a}}(r^\prime) =
\sum_j \int\,dr^\prime M_{ij}(r, r^\prime) \hat{a}_j(r^\prime).
\end{equation}
%
We can formally solve our linearized equation of motion,
%
\begin{equation}
    \label{eq:formal_sol}
    \hat{\mathbf{a}}(r, t) = e^{\mathbf{M}(r,
r^\prime)t}\ast\hat{\mathbf{a}}(r^\prime, 0) + \int_0^t d\tau\,
e^{\mathbf{M}(r, r^\prime)(t-\tau)} \ast \boldsymbol{\xi} (r^\prime, \tau),
\end{equation}
%
And use this formal solution to evaluate the dynamic pair correlation function,
%
\begin{align} 
    \label{eq:formal_corr}
    \mean{\hat{\mathbf{a}}(r,t)\hat{\mathbf{a}}^\dagger(r^\prime, t^\prime)} &=
             e^{\mathbf{M}(r, r_1)t}
        \ast \mean{\hat{\mathbf{a}}(r_1, 0)\hat{\mathbf{a}}^\dagger(r_2, 0)}
        \ast e^{\mathbf{M}^\dagger(r^\prime, r_2)t^\prime} \nonumber \\ 
     &+ \int_0^t \int_0^{t^\prime}d\tau d\tau^\prime\, 
        e^{\mathbf{M}(r, r_1)(t-\tau)} 
        \ast \mean{ \boldsymbol{\xi}(r_1, 0)\boldsymbol{\xi}^\dagger(r_2, 0)}
        \ast e^{\mathbf{M}^\dagger(r^\prime, r_2)(t^\prime-\tau^\prime)}.
\end{align}
%
To evaluate the equilibrium correlation function we take the limit as each time
goes to infinity together ($t = t^\prime \rightarrow \infty$). It is important
to note that every eigenvalue of $\mathbf{M}$ must be negative for our solution
to decay to equilibrium in the long time limit (eg.
$lim_{t\rightarrow\infty}\hat{\mathbf{a}}(r, t) = 0$) and as such the first
term in equation \ref{eq:formal_corr} won't contribute to the equilibrium
pair correlation. This is as we might expect as the first term holds the
contributions to the dynamic correlation function from the initial conditions.
The second term can be evalutated by substituting the noise correlation from
equation \ref{eq:noise_form} and
evaluating the delta function.
%
\begin{equation}
    \mathbf{\Gamma}(r, r^\prime) = \lim_{t\rightarrow\infty}
    \mean{\hat{\mathbf{a}}(r, t)\hat{\mathbf{a}}^\dagger(r^\prime, t)} =
    \int_0^\infty dz \,e^{\mathbf{M}(r, r_1)z}\ast\mathbf{L}(r_1, r_2)\ast
    e^{\mathbf{M}^\dagger(r^\prime, r_2)z}
\end{equation}
%
Considering the product $\mathbf{M}(r, r_1)\ast\mathbf{\Gamma}(r_1, r^\prime)$
and performing an integration by parts yields the final generalized Einstein
relation.
%
\begin{equation}
    \label{eq:generalized_einst} 
    \mathbf{L}(r, r^\prime) = - \l\lbrace
          \mathbf{M}(r, r_1) 
     \ast \mathbf{\Gamma}(r_1, r^\prime) 
        + \mathbf{\Gamma}(r, r_1)
     \ast \mathbf{M}^\dagger(r_1, r^\prime) \r\rbrace
\end{equation}
%

As we can see from equation \ref{eq:generalized_einst}, near equilibrium the
noise correlation function is a simple function of the pair correlation
function, $\mathbf{\Gamma}(r, r^\prime)$ and the linearized transport
coefficient $\mathbf{M}(r, r^\prime)$.

As a simple check we apply our result to the original work of Einstein. Recall
that in the over damped limit the equation of motion for the velocity for a 1
dimensional Brownian particle is,
%
\begin{equation}
    \f{\partial v(t)}{\partial t} = -\gamma v (t) + \xi(t).
\end{equation}
%
This equation is alread linear so we can pick off the linearized transport
coefficient as $-\gamma$. The pair correlation function in equilibrium is given
by equipartition theorem as,
%
\begin{equation}
    \mean{v^2} = \f{k_b T}{m}.
\end{equation}
%
Simply applying equation \ref{eq:generalized_einst} we find,
%
\begin{equation}
    \mean{\xi(t)\xi(t^\prime)} = 2 \f{k_b T \gamma}{m} \d(t - t^\prime),
\end{equation}
%
As expected. Satisfied that equation \ref{eq:generalized_einst} reduces to 
the correct result for the base case we proceed to examine two examples
that are guininely nonlinear field theories.

%%%%%%%%%%%%%%%%%%%%%%%%%%%%%%%
\section{Example 1 - Model A} %
%%%%%%%%%%%%%%%%%%%%%%%%%%%%%%%

As a first nontrivial example of calculating an Einstein relation consider the
following free energy functional under non-conservative, dissipative dynamics.
%
\begin{gather}
    \beta \F[\phi] = \int dr \l\lbrace \f{1}{2}\vert \nabla
    \phi(x) \vert^2 + \f{r}{2}\phi^2(x) + \f{u}{4!}\phi^4(x)  +
    h(x)\phi(x)\right\rbrace \\ \f{\partial \phi(x,t)}{\partial t} = -\Gamma
    \left(\f{\delta \beta \F[\phi]}{\delta \phi(x)}\right) + \xi(x, t)
\end{gather}
%
The random driving force, $\xi$, is Gaussian noise, uncorrelated in time.
%
\begin{align} \l\langle \xi (x, t) \r\rangle &= 0 \\ \l\langle \xi (x, t)
\xi(x^\prime, t^\prime) \r\rangle  &= L(x-x^\prime) \delta (t - t^\prime)
\end{align}
%
To compute the Einstein relation for this theory we start by calculating the
pair correlation function using the equilibrium partition function and
Boltzmann statistics.

\subsection{The partition function route}

In equilibrium the probability of particular field configuration is given by
the Boltzmann distribution.
%
\begin{equation} \mathcal{P}_{eq}[\phi] =
\f{e^{-\beta\F[\phi]}}{\mathcal{Z}[h(x)]} \end{equation}
%
Where, $\mathcal{Z}[h(x)]$ is the partition functional and is given by a path
integral over all field configurations.
%
\begin{equation} \mathcal{Z}[h(x)] = \int \mathcal{D}[\phi] e^{-\beta\F[\phi]}
\end{equation}
%
Evaluation of the partition function is of some importance because it plays the
role of a moment generating function.
%
\begin{equation}\label{gen} \f{1}{\Z[h]}\f{\delta^n \Z[h]}{\delta
h(x_1)...\delta h(x_n)} = \langle \phi(x_1)...\phi(x_n)\rangle \end{equation}
%
In general the partition function cannot be computed directly, but in the
special case of Gaussian free energies it can. To that end we consider
expanding $\phi$ around an equilibrium solution, $\phi(x) = \phi_0 +
\Delta\phi(x)$, and keeping terms to quadratic order in the free energy.
%
\begin{equation} \beta\F[\Delta\phi] = \int dr \,\left\lbrace
\f{1}{2}\Delta\phi(x) \left(r - \nabla^2 + \f{u}{2}\phi_0^2\right)
\Delta\phi(x) - h(x)\Delta\phi(x) \right\rbrace \end{equation}
%
Here the partition function is written in a suggestive form. As stated
previously, functional integrals are difficult to compute in general, but
Gaussian functional integrals do have a solution.

\subsubsection{Computing the Pair correlation function in the Gaussian
approximation}

To compute the pair correlation function we use the Fourier space variant of
the partition function,
%
\begin{equation} 
    \Z[\tilde{h}(k)] \propto \exp\left\lbrace \f{1}{2}\int dk\,
        \f{h(k)h^{*}(k)}{r + \f{u}{2}\phi_0^2 +  \vert k \vert^2}
        \right\rbrace.
\end{equation} 
%
The pair correlation function, $\langle
\Delta\tilde{\phi}(k)\Delta\tilde{\phi}^{*}(k)\rangle$, is then computed using
equation \ref{gen}.
%
\begin{equation} \l\langle \Delta\fphi(k)\Delta\fphi^{*}(k^\prime) \r\rangle =
\f{2\pi \delta(k+k^\prime)}{r + \f{u}{2}\phi_0^2 + \vert k \vert^2}
\end{equation}
%
%%%%%%%%%%%%%%%%%%%%%%%%%%%%%%%%%%%%%%%%%%%
\subsection{The Equation of Motion Route} %
%%%%%%%%%%%%%%%%%%%%%%%%%%%%%%%%%%%%%%%%%%%

The equation of motion supplies a second method for evaluating the pair
correlation function in equilibrium.
%
\begin{equation} \f{\partial \phi}{\partial t} =
-\Gamma\left((r-\nabla^2)\phi(x,t) + \f{u}{3!}\phi^3(x,t)\right) + \xi(x, t),
\end{equation}
%
Our equation of motion, can be linearized around an equilibrium solution,
$\phi_0$, just as we did in the partition function route to the pair
correlation function. In a similar vain, we will Fourier transform the equation
of motion as well.
%
\begin{equation} \f{\partial \Delta\fphi(k, t)}{\partial t} = -\Gamma\left((r +
\f{u}{2}\phi_0 + \vert k \vert^2)\Delta\fphi(k,t)\right) + \xi(x,t)
\end{equation}
%
Comparing with our generalized approach we can read of $M(k, k^\prime)$ from
the lineared equation of motion:
%
\begin{equation} M(k, k^\prime) = -\Gamma\l((r + \f{u}{2}\phi_0 + \vert k
\vert^2)\r)\d(k + k^\prime) \end{equation}
%
Finally, once we compute the generalized Einstein relation with our specific
pair correlation and $M(k, k^\prime)$ we find,
%
\begin{equation} L(k, k^\prime) = 2\Gamma \d(k + k^\prime), \end{equation}

Or equivalently,

\begin{equation} L(x, x^\prime) = 2\Gamma \d(x - x^\prime).  \end{equation}

%%%%%%%%%%%%%%%%%%%%%%%%%%%%%%%%%%%%%%%%%%%%%%%%%%%%%%%%%%%%%%%%
\section{Example 2 - Time Dependent Density Functional Theory} %
%%%%%%%%%%%%%%%%%%%%%%%%%%%%%%%%%%%%%%%%%%%%%%%%%%%%%%%%%%%%%%%%

In time dependent density functional theory (TDDFT) we have an equation of
motion of the following form,

\begin{equation} \f{\partial \rho(r, t)}{\partial t} = D_0 \nabla \cdot
\l[\rho(r,t)\nabla \l(\f{\d \F[\rho]}{\d \rho}\r)\r] + \xi(r, t) \end{equation}
%
Where, $D_0$ is the equilibrium diffusion constant and $\xi$ is the stochastic
driving force. We assume once again that the driving force has no bias, but we
now allow the noise strength to be a generic kernel $L(r, r^\prime)$.
%
\begin{align} \langle \xi(r,t) \rangle &= 0 \\ \l\langle \xi(r, t)
\xi(r^\prime, t^\prime) \r\rangle &= L(r, r^\prime) \d (t -t^\prime)
\end{align}
%
%%%%%%%%%%%%%%%%%%%%%%%%%%%%%%%%%%%%%%%%%%%%%%%%%%%%%%%%%%%%%
\subsection{Pair Correlation from the Partition Functional} %
%%%%%%%%%%%%%%%%%%%%%%%%%%%%%%%%%%%%%%%%%%%%%%%%%%%%%%%%%%%%%

Just like with the $\phi^4$ model we want to expand our free energy functional
around an equilibrium solution. In this case our free energy functional is
generic so this expansion is purely formal.
%
\begin{equation} \F[\rho] = \F_{eq} + \beta\int dr \l(\l.\f{\d \F[\rho]}{\d
\rho(r)}\r)\r\vert_{\rho_{eq}}\Delta\rho(r) + \f{1}{2} \int dr \int dr^\prime
\Delta\rho(r) \l(\l.\f{\d^2 \F[\rho]}{\d \rho(r) \d
\rho(r^\prime)}\r)\r\vert_{\rho_{eq}} \Delta\rho(r^\prime) \end{equation}
%
The first term we can neglect as it adds an overall scale to the partition
function that will not affect any of moments. Second moment only shifts the
average so we can ignore it as well and so we're left with a simple quadratic
free energy once again.
%
\begin{equation} \F[\rho] = \f{1}{2}\int dr \int dr^\prime \Delta \rho(r)
\Gamma^{-1}(r, r^\prime) \Delta \rho(r^\prime) \end{equation}
%
Where, $\Gamma^{-1}(r, r^\prime)$ is the second functional derivative of the
free energy functional in equilibrium. Computing the pair correlation function
from the partition function yields, as might be expected,
%
\begin{equation} \l\langle \Delta\rho(r) \Delta\rho(r^\prime) \r\rangle =
\Gamma(r, r^\prime) \end{equation}
%
%%%%%%%%%%%%%%%%%%%%%%%%%%%%%%%%%%%%%%%%%%%%%%%
\subsection{Linearing the equation of motion} %
%%%%%%%%%%%%%%%%%%%%%%%%%%%%%%%%%%%%%%%%%%%%%%%

Linearizing the equation of motion about an equilibrium solution we find the
following form,
%
\begin{equation} \f{\partial \Delta \rho (r, t)}{\partial t} = D_0\nabla \cdot
\l[\rho_{eq}(r)\nabla \l(\Gamma^{-1}(r, r^\prime)\ast \Delta\rho(r^\prime,
t)\r)\r] + \xi(r, t) \end{equation}
%
Once again we can read of the kernel $M(r, r^\prime)$ from the linearized
equation.
%
\begin{equation} M(r, r\prime) = D_0\nabla \cdot \l[\rho_{eq}(r)\nabla
\l(\Gamma^{-1}(r, r^\prime)\r)\r] \end{equation}
%
Plugging into the generalized Einstein relation, we find a the factors of the
pair correlation cancel giving a simple form for the kernel $L(r, r^\prime)$.
%
\begin{equation} L(r, r^\prime) =
-2D_0\nabla\cdot\l(\rho_{eq}(r)\nabla\r)\delta(r - r^\prime) \end{equation}
%

\nocite{Ronis, Fox_and_Uhlenbeck, Lax}
