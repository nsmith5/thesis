\label{dft_of_freezing}

The classical density functional theories derived in chapter
\ref{fundamentals} was first established to study inhomogenous fluids.
Interestingly, one can think of the solid state as an especially extreme
case of an inhomogeneous fluid \cite{HANSEN-CH6}.  In this case, we can
use CDFT to study the circumstances under which the density field develops
long range periodic solutions (ie., solidification).  While not expressed
in precisely this language, this idea dates back as far as 1941 with the
early work of Kirkwood and Monroe \cite{KIRKWOOD_MONROE41} and was later
significantly refined by Youssof and Ramakrishnan \cite{RAMAKRISHNAN79}.

%%%%%%%%%%%%%%%%%%%%%%%%%%%%%%%%
\section{Amplitude Expansions} %
%%%%%%%%%%%%%%%%%%%%%%%%%%%%%%%%

More precisely, if we would like to study the liquid-solid transitions we
can expand the density field in a set of plane-waves that span the unit
cell of the solid phase,
%
\begin{equation} \rho(\mathbf{x}) = \f{1}{V_{cell}} \sum_i \xi_i e^{i
\mathbf{k}_i \cdot \mathbf{x}}.  \end{equation}
%
In the solid phase, the amplitudes $\xi_i$ become nonzero for
$\mathbf{k}_i \ne 0$ and describe the periodic structure of the solid
density field.  In the liquid, only the $\mathbf{k}_i = 0$ value remains
(ie., the density becomes constant).  Since the density profile can be
reconstructed with knowledge of all the of amplitudes we can rewrite the
grand potential as a function of the amplitudes only, 
%
\begin{equation} \Omega[\rho(x)] = \Omega(\lbrace\xi_i\rbrace),
\end{equation}
%
From which the equilibrium condition becomes,
%
\begin{equation} \f{\partial \Omega(\lbrace\xi_i\rbrace)}{\partial \xi_k}
= 0, \,\,\forall \,k.  \end{equation}
%

Using this technique and the approximate free energy of Equation
\ref{cdft_free_energy} results in good agreement with experiment can be found
for equilibrium properties such as the density change of solidification.
{\color{ForestGreen} table of results from Ramakrishnan? } The implication of
these results and their success in predicting equilibrium thermodynamic
properties points a the amplitudes, $\xi_i$, as natural and appropriate order
parameters from the problem of solidification. 

In spite of these successes, CDFT has a major short coming in that it is an
equilibrium theory. This short coming cannot be understated in any attempt to
be a theory of the mechanical and thermodynamic properties of metal systems.
While some mechanical properties are well understood with an understanding of
the perfect crystalline state, such as the Young's modulus, the vast majority
of mechanical properties of solids is due to their microstructural defects.

Artifacts such as grains boundaries, dislocations and vacancies play key roles
not only in the mechanical properties but also the kinetic pathways of certain
phase transformations. For instance, dislocations can act to catalyze
precipication in binary alloys [cite Vahid].

The problem that we face is then, how do we examine these defects and
microstructural elements but retain some of the positive aspects of CDFT? More
specifically, what path does density profile take to equilibrium?

%%%%%%%%%%%%%%%%%%%%%%%%%%%%%%%%%%%%%%%%%%%%%
\section{Dynamic Density Functional Theory} %
%%%%%%%%%%%%%%%%%%%%%%%%%%%%%%%%%%%%%%%%%%%%%

We can extend our density functional theory to a dynamical model using
techniques from non-equilibrium statistical mechanics. We begin by considering
some non-equilibrium probability distribution over phase space, $f(\q, \p, t)$,
and its equation of motion,
%
\begin{equation}
    \label{liouville}
    \f{\partial f(\q, \p, t)}{\partial t} = -\l\lbrace f, \mathcal{H} \r\rbrace.
\end{equation}
%
Where, $\l\lbrace \cdot, \cdot \r\rbrace$, denotes the Poisson bracket,
%
\begin{equation}
    \l\lbrace f, g \r\rbrace = \sum_{i = 0}^N
\end{equation}

{\color{ForestGreen} { \bfseries
    
    % Basic idea of this section:
    
    Here we need to make the point that, while equilibrium results are
    important, a huge amount of mechanical properties depend on
    nonequilibrium features such as dislocations, grain boundaries etc.
    Additionally, these things really depend on the pathway we take to
    equilibrium. 

    This is where the free energy functional used by Ramakrishnan as
    significant set backs. Deriving equations of motion is fine so we can
    talk about over damped hydrodynamics here and the projection operator
    stuff. We can also discuss Archer's work and the distinction between
    the Langevin equations for the operator versus the mean transport
    equation.

    The problem is that the density profile in equilibrium is essentially
    a sum of sharply peaked gaussians that are hard to resolve
    numerically. Want to probe the dislocation dynamics? Best to find a
    better behaved theory.} }

%%%%%%%%%%%%%%%%%%%%%%%%%%%%%%%%%%%%%%
\section{Phase Field Crystal Theory} %
%%%%%%%%%%%%%%%%%%%%%%%%%%%%%%%%%%%%%%

{\color{ForestGreen} {\bfseries

% Basic idea of this chapter

Introduce PFC. The essential physics of solidifications is retained,
and much larger length and timescales can be explored. The assumption
here is that we haven't lost the qualitative mechanisms we're
interested even if we lost some quantitative agreement with theory.} }
