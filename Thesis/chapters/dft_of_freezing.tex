The classical density functional theories derived in chapter \ref{fundamentals} was first established to study inhomogenous fluids.
Interestingly, one can think of the solid state as an especially extreme case of an inhomogeneous fluid \cite{HANSEN-CH6}.
In this case, we can use CDFT to study the circumstances under which the density field develops long range periodic solutions (ie., solidification).
While not expressed in precisely this language, this idea dates back as far as 1941 with the early work of Kirkwood and Monroe \cite{KIRKWOOD_MONROE41} and was later significantly refined by Youssof and Ramakrishnan \cite{RAMAKRISHNAN79}.

More precisely, if we would like to study the liquid-solid transitions we can expand the density field in a set of plane-waves that span the unit cell of the solid phase,
%
\begin{equation}
    \rho(\mathbf{x}) = \f{1}{V_{cell}} \sum_i \xi_i e^{i \mathbf{k}_i \cdot \mathbf{x}}.
\end{equation}
%
In the solid phase, the amplitudes $\xi_i$ become nonzero for $\mathbf{k}_i \ne 0$ and describe the periodic structure of the solid density field while in the liquid, only the $\mathbf{k}_i = 0$ value remains (ie., the density becomes constant).
Since the density profile can be reconstructed with knowledge of all the of amplitudes we can rewrite the grand potential as a function of the amplitudes only, 
%
\begin{equation}
    \Omega[\rho(x)] = \Omega(\lbrace\xi_i\rbrace),
\end{equation}
%
Whence the equilibrium condition becomes,
%
\begin{equation}
    \f{\partial \Omega(\lbrace\xi_i\rbrace)}{\partial \xi_k} = 0, \,\,\forall \,k.
\end{equation}

%%%%%%%%%%%%%%%%%%%%
\section{Dynamics} %
%%%%%%%%%%%%%%%%%%%%