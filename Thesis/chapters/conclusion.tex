\label{chapter:conclusion}

% Restate the research goals and aims

The goal of the current research was threefold: The first two goals were to
present two extensions to the binary XPFC theory. The first extension we
presented was the addition of an enthalpy of mixing to account for non-ideal
mixing and second was a general phenomenology for modelling density pair
correlation functions. The final goal was to apply these improvements to the
study of multi-step nucleation pathways in precipitation. 

% Reiterate the fashion in which these goals were fulfilled

As seen in Chapter \ref{chapter:improvements} we saw these two extensions
derived in detail and explored the new landscape of equilibrium phase diagrams
they result in. We also noted that metastable artifacts of the phase diagram
can be reproduced such a submerged metastable liquid spinodal below the
eutectic point in a eutectic material. In Chapter \ref{chapter:applications} we
constructed a simplified model of precipitation and showed that a submerged
metastable liquid spinodal can result in a multi-step nucleation pathway
observed experimentally in silver and gold nanoparticles \cite{LOH17}. 

% Evaluate the current situation (show that the gap has closed)

We feel that these two improvements are an important step in the constructing
a robust and complete theoretical picture of binary alloys.

% Talk vaguely about broader impact of these changes 
