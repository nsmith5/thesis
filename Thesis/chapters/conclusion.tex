\label{chapter:conclusion}

The goal of the current research was threefold: The first two goals were to
present two extensions to the binary XPFC theory. The first extension we
presented was the addition of an enthalpy of mixing to account for non-ideal
mixing and second was a general phenomenology for modelling density pair
correlation functions. In Chapter \ref{chapter:improvements} we saw these two
extensions derived in detail and explored the new landscape of equilibrium
phase diagrams they result in. We also noted that metastable artifacts of the
phase diagram can be reproduced such a submerged metastable liquid spinodal
below the eutectic point in a eutectic material. 

The final goal was to apply these improvements to the study of multi-step
nucleation pathways in precipitation. In Chapter \ref{chapter:applications} we
constructed a simplified model of precipitation and showed that a submerged
metastable liquid spinodal can indeed result in a multi-step nucleation pathway
like those observed experimentally in silver and gold nanoparticles
\cite{LOH17}. 

Beyond the study of precipitation there are other processes in which our
improvements to the XPFC alloy model may prove insightful. Investigating the
effects of elasticity on monotectic and  syntectic nucleation and growth are
one clear direction that can now be studied given the capacity to model the
equilibrium phase diagrams and kinetics of these materials. As previously
noted, the problem of stability of nanocrystalline binary alloys is another
topic of importance that has  been shown to depend on the enthalpy of mixing of
the system \cite{MURDOCH13}.
