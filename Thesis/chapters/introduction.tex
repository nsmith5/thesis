\label{chapter:introduction}

% Thesis introduction outline: https://student.unsw.edu.au/introductions

% 1) State the general topic and give some background

The study of alloys in materials physics is a pursuit of incredibly broad
impact since the functional properties of materials depend on their
microstructure, which forms through non-equilibrium phase transformations
during the process of forming a material. As a result, industries  as diverse
as those dealing in commercial products based on steel and aluminium to the
burgeoning  markets dealing in nano-fabrication and optoelectronics are
affected by research in alloy materials.  One of the most useful paradigms for
understanding complex alloy microstructure is that of the binary alloy.

One surprising aspect of binary alloys is the rich diversity of properties and
behaviours they display. Because material properties depend on the
microstructural details of the material, they have a strong processing path
dependence.  Grain boundaries, vacancies, dislocations and other
microstructural artifacts are all intimately tied to the manufacturing process
of the alloy. This means that the study of solids can never be completely
separated from the study of solidification. As such, the diversity of material
properties and behaviours we see in binary alloys can be directly attributed to
the diversity of processes for their construction.

Given the importance of binary systems, it is critical to construct models that
can explain the diversity of behaviour we see in them. At the moment, models of
alloy solidification can be categorized by the length and time scales they
accurately describe. At macroscopic length and time scales, we have continuum
methods of heat and mass transport and associated finite element methods of
analysis. These methods are appropriate for studying large castings, for
example. On length scales $\mathcal{O}(10^{-6}m)$ to $\mathcal{O}(10^{-3}m)$ we
use \textit{Phase Field} methods to study phenomena such as dendritic growth
and chemical segregation. On still finer length scales from
$\mathcal{O}(10^{-9}m)$ to $\mathcal{O}(10^{-6}m)$ and on relatively long
timescales, we have the methods of \textit{Phase Field Crystal} (PFC) theory
and dislocation dynamics.  These methods are appropriate for studying
nanoscopic changes that occur on diffusive timescales such as dislocation
motion, creep, grain boundary motion and micro segregation. At a still finer
scale and on very short time scales ($\mathcal{O}(10^{-12}s)$) we have the
methods of molecular dynamics and density functional theory. These methods are
appropriate for the study of transport coefficients and interaction potentials.

% 2) Outline the current situation

In this thesis, we'll focus on extending a branch of binary PFC theory known as
the binary {\it XPFC} model --where the "X" in XPFC signifies a class of PFC
models  constructed to controllably simulate a robust range of metallic and
non-metallic crystal symmetries compared to the original PFC models. PFC binary
models have been successful in describing a broad selection of phenomena in
binary alloys.  These successes include eutectic and dendritic solidification
\cite{ELDER07}, the Kirkendall effect \cite{ELDER11_KIRKENDALL, LU15}, solute
drag \cite{GREENWOOD12}, clustering and precipitation \cite{FALLAH12, FALLAH13,
FALLAH13_AlCu_experiment}, colloidal ordering in drying suspensions
\cite{GANAI13}, epitaxial growth and island formation \cite{ELDER10_NANOISLAND,
LU16}, and ordered crystals \cite{ALSTER17} to name a few. 

The PFC theory is derived from Classical Density Functional Theory (CDFT) and
as such, it can be considered a simplified density functional theory.  In
practice, two different variants of the PFC theory are used, as alluded to
above: the original model developed by Elder \textit{et al} \cite{ELDER07} and
the Structural Phase Field Crystal (XPFC) model developed by Greenwood
\textit{et al.} \cite{GREENWOOD11_BINARY}.

The original model was the first PFC theory of binary alloys and contains some
important physical properties of binary alloys. However, it is a very reduced
form of CDFT and it therefore lacks completeness in its ability to describe
binary alloys. Specifically, the original model uses an expansion in
concentration that is actually a density difference not a concentration, and
the model has a limited ability to describe a realistic or robust range of
phase diagrams. The original model also uses a very simplified correlation
kernel which limits its ability to describe a variety of crystal lattice
structures.

The XPFC model is an improvement that ameliorates the above problems. The
concentration is left unexpanded allowing for construction of realistic global
phase diagrams instead of local expansions. More significantly, the XPFC model
provides a phenomenology for modelling two-point correlation functions that
succeeded in describing solidification of a variety of lattice structures, as
well as transformations between different crystal lattices. Simplifications of
the multi-modal approach first introduced with the XPFC formalism has been used
to produce hexagonal, square, kagome, honeycomb, rectangular and other lattices
in 2 dimensions \cite{MKHONTA13}.

% 3) Evaluate the current situation and identify a gap

In introducing its phenomenology for modelling correlation functions, the
binary XFPC theory tacitly assumes that there is some preferred structure at
high concentration and some other structure preferred at low concentration.
This assumption can be limiting in situations that have a specific crystalline
structure at intermediate concentrations, such as materials with a syntectic
phase diagram. At the syntectic point a solid of intermediate concentration
solidifies along the interface between a solute rich and solute poor liquid.
The XPFC model also assumes no long wavelength correlations in the
concentration field which, in practice, means the model has an ideal free
energy of mixing.  This is another limitation of the XPFC model because the
enthalpy of mixing is not generally zero for alloy systems.

% 5) State the research goals and aims

The goal of the current research is threefold: The first two goals are to
present two important improvements to the binary XPFC theory. The first
improvement is a more general phenomenology for modelling pair correlation
functions of a binary material. The second improvement is to extend the free
energy of mixing beyond ideality to account for circumstances when the enthalpy of
mixing is not negligible. The third goal is to use the new XPFC model derived
herein to the elucidate the multi-step nucleation process seen in certain
diffusion-limited systems including gold and silver nanoparticles \cite{LOH17}.

% 6) Outline the order of information

The remainder of this thesis is divided into 5 chapters:
%
\begin{description}
    \item [Chapter \ref{chapter:cdft_intro}] { Classical Density Functional
        Theory (CDFT) is introduced and derived from fundamental principles of
        quantum statistical mechanics.
    }
    \item [Chapter \ref{chapter:cdft_of_freezing}] { CDFT theory of
        solidification is described and discussed. The density functional
        theory is extended to a dynamic, non-equilibrium theory, and the Phase
        Field Crystal (PFC) Theory is introduced from it as a simplified
        density functional theory.
    }
    \item [Chapter \ref{chapter:binary}] {Binary PFC theory is derived and
        previous simplified alloy PFC models are summarized and discussed.
    }
    \item [Chapter \ref{chapter:improvements}] {Improvements to the XPFC binary
        alloy theory are derived. This chapter contains novel contributions to
        the field.
    }
    \item [Chapter \ref{chapter:applications}] {The new XPFC alloy model
        derived herein is applied to model to the problem of multi-step
        nucleation of nanoparticles from solution in diffusion limited systems
        and potential future applications of this model are discussed.
    } 
\end{description}
%
