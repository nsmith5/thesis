\label{chapter:introduction}

% Thesis introduction outline: https://student.unsw.edu.au/introductions

% 1) State the general topic and give some background


% 2) Outline the current situation


% 3) Evaluate the current situation and identify a gap


% 4) Identify the importance of the research (the size of the gap)


% 5) State the research goals and aims


% 6) Outline the order of information
This thesis is divided into 6 chapters. Chapter \ref{fundamentals} introduces
Classical Density Functional theory (CDFT) for the study of inhomogeneous
liquids. Chapter \ref{dft_of_freezing} shows how we can use CDFT to study
solidification, extends CDFT to a dynamic non-equilibrium theory and introduces
a simplified density function theory; the Phase Field Crystal (PFC) theory.
Chapter \ref{chapter:binary_pfc} establishes a binary version of the PFC theory and
summarizes the previous simplified binary PFC models including the current
binary structural phase field crystal (XPFC) model. Chapter
\ref{xpfc_improvements} discusses improvements to the XPFC model and contains
novel contribution to the field. Chapter \ref{applications} concludes the
thesis by applying the improved XPFC model to the problem of multistep
nucleation of nanoparticles and discusses potential future applications.
