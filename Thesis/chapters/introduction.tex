\label{chapter:introduction}

% Thesis introduction outline: https://student.unsw.edu.au/introductions

% 1) State the general topic and give some background

The study of two binary alloys in materials physics is a pursuit of incredibly
broad impact. Industries as distant as the large commerial materials such as
steel and aluminium producers and burgeoning markets of nano-fabrication and
optoelectronics are affected by research in binary alloys.

One suprising aspect of binary alloys is the rich diversity of properties and
behaviours they display. Because material properties depend on the
microstructural details of the material they have a strong path dependence.
Grain boundaries, vacancies, dislocations and other microstructural are all
intimitly tied to the manufactoring process of the alloy. This means that the
study of solids can never be completely seperated from the study of
solidification. As such the diversity of material properties and behaviours we
see in binay alloys can be directly attributed to the diversity of processes
for their construction.

Given the importance of these systems it is important to construct models that
can explain diversity of behavior we see in them. At the moment models of
solidification can be categorized by the length and time scales they accurately
discribe. The the macroscopic length and time scales we have continuum methods
of heat and mass transport and associated finite element methods of analysis.
These methods are appropriate for studying large castings for example. On
length scales $\mathcal{O}(10^{-3})m$ to $\mathcal{O}(10^{-6})$ we use
\textit{Phase Field} methods to study phenomena such as dendritic growth and
chemical segregation. On still finer length scales from $\mathcal{O}(10^{-9})$
to $\mathcal{O}(10^{-6})$ and on relatively long timescales we have the methods
of \textit{Phase Field Crystal} theory and dislocation dynamics. These methods
are appropriate for studying nanoscopic changes that occur on diffusive
timescales such as dislocation motion, creep, grain boundary motion and micro
segregation. At a still finer scale and on very short time scales
($\mathcal{O}(10^{-12}s)$) we have the methods of molecular dynamics and
density functional theory. This methods are appropriate for the study of
transport coefficients and interaction potentials.

% 2) Outline the current situation

In this thesis we'll focus on the binary Phase Field Crystal (PFC) theory. The 
binary phase field crystal theory has been successful in describing a broad
selection of phenomena in binary alloys. These successes include the Kirkendall
effect \cite{ELDER11_KIRKENDALL, LU15}, solute drag \cite{GREENWOOD12}, 
clustering and precipitation \cite{FALLAH12, FALLAH13, FALLAH13_AlCu_experiment},
colloidal ordering in drying suspensions \cite{GANAI13}, epitaxial growth
and island formation \cite{ELDER10_NANOISLAND, LU16}, and ordered crystals
\cite{ALSTER17} to name a few. 

The PFC theory is derived from Classical Density Functional Theory (CDFT) and as
such it is a sort of simplified density functional theory. In practice,
two different variants of the PFC theory are used in practice: The original
model developed by Elder \textit{et al} \cite{ELDER07} and the Structural
Phase Field Crystal (XPFC) model developed by Greenwood \textit{et al.}
\cite{GREENWOOD11_BINARY}. The original model is a very reduced form of
CDFT and so it lacks completeness in its ability to describe binary alloys.
Specifically, the original model uses an expansion in concentration that limits 
its ability to describe a realistic phase diagram. The original model also
uses a very simplified correlation kernel which limits its ability to describe
a variety of crystal lattice structures. The XPFC model is an improvement 
in that is ameliorates both of these problems. The concentration is left 
unexpanded allowing for construction of realistic global phase diagrams instead
of local expansions. The XPFC model provided a phenomenology for modelling
correlation functions that succeeded in describing solidification of a
variety of lattice structures ({\color{ForestGreen} Find that paper about
solidification of all 2d lattices}). 

% 3) Evaluate the current situation and identify a gap

In introducing its phenomenology for modelling correlation function, the 
XFPC theory tacitly assumes that there is some preferred structure at high
concentration and some other structure preferred at low concentration. This
can be limited in situation that have a specific structure specifically at intermediate
concentrations such a syntectic material. The XPFC model also assumes no
long wavelength correlations in the concentration field which in practice means
the model has an ideal free energy of mixing. This is another limitation of 
the XPFC model because in general the enthalpy of mixing is not zero for 
binary alloys.

% 5) State the research goals and aims

This goal of the current research is to present two improvements to the
binary XPFC theory. The first improvement is a more general phenomenology
for modelling pair correlation functions of a material. The second 
improvement is two extend the free energy of mixing beyond ideality to 
account for circumstances when the heat of mixing is not negligible.

% 6) Outline the order of information

This thesis is divided into 6 chapters:
%
\begin{description}
    \item [Chapter \ref{chapter:cdft_intro}] { Classical Density Functional
        Theory (CDFT) is introduced and derived from fundamental principles of
        quantum statistical mechanics.
    }
    \item [Chapter \ref{chapter:cdft_of_freezing}] { CDFT theory of
        solidification is described and discussed. The density functional
        theory is extended to a dynamic, non-equilibrium theory and the Phase
        Field Crystal (PFC) Theory is introduced as a simplified density
        functional theory.
    }
    \item [Chapter \ref{chapter:binary}] { Binary PFC theory is established and
        previous simplified models are summarized and discussed.
    }
    \item [Chapter \ref{chapter:improvements}] { Improvements to the XPFC model
        are discussed and contains novel contribution to the field.
    }
    \item [Chapter \ref{chapter:applications}] { Concludes the thesis by
        applying the improved XPFC model to the problem of multistep nucleation
        of nanoparticles and discusses potential future applications.
    } 
\end{description}
%

