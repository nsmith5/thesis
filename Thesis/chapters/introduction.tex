\label{chapter:introduction}

% Thesis introduction outline: https://student.unsw.edu.au/introductions

% 1) State the general topic and give some background


% 2) Outline the current situation


% 3) Evaluate the current situation and identify a gap


% 4) Identify the importance of the research (the size of the gap)


% 5) State the research goals and aims


% 6) Outline the order of information
This thesis is divided into 6 chapters:
%
\begin{description}
    \item [Chapter \ref{fundamentals}] { 
Classical Density Functional Theory (CDFT) is introduced and derived
from fundamental principles of quantum statistical mechanics.
    }
    \item [Chapter \ref{dft_of_freezing}] {
CDFT theory of solidification is described and discussed. The density
functional theory is extended to a dynamic, non-equilibrium theory 
and the Phase Field Crystal (PFC) Theory is introduced as a simplified
density functional theory.
    }
    \item [Chapter \ref{chapter_pfc}] {
Chapter \ref{chapter:binary_pfc} establishes a binary version of the PFC theory
and summarizes the previous simplified binary PFC models including the current
binary structural phase field crystal (XPFC) model.
    }
    \item [Chapter \ref{}] {
Chapter \ref{xpfc_improvements} discusses improvements to the XPFC model and
contains novel contribution to the field.
    }
    \item [Chapter \ref{applications}] {
Chapter \ref{applications} concludes the thesis by applying the improved XPFC
model to the problem of multistep nucleation of nanoparticles and discusses
potential future applications.
    } 
\end{description}
%

