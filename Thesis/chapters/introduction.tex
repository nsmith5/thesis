\label{chapter:introduction}

% Thesis introduction outline: https://student.unsw.edu.au/introductions

% 1) State the general topic and give some background


% 2) Outline the current situation


% 3) Evaluate the current situation and identify a gap


% 4) Identify the importance of the research (the size of the gap)


% 5) State the research goals and aims

This goal of the current research is to present two improvements to the
binary XPFC theory. The first improvement is a more general phenomenology
for modelling pair correlation functions of a material. The second 
improvement is two extend the free energy of mixing beyond ideality to 
account for circumstances when the heat of mixing is not negligible.

% 6) Outline the order of information

This thesis is divided into 6 chapters:
%
\begin{description}
    \item [Chapter \ref{chapter:cdft_intro}] { Classical Density Functional
        Theory (CDFT) is introduced and derived from fundamental principles of
        quantum statistical mechanics.
    }
    \item [Chapter \ref{chapter:cdft_of_freezing}] { CDFT theory of
        solidification is described and discussed. The density functional
        theory is extended to a dynamic, non-equilibrium theory and the Phase
        Field Crystal (PFC) Theory is introduced as a simplified density
        functional theory.
    }
    \item [Chapter \ref{chapter:binary}] { Binary PFC theory is established and
        previous simplified models are summarized and discussed.
    }
    \item [Chapter \ref{chapter:improvements}] { Improvements to the XPFC model
        are discussed and contains novel contribution to the field.
    }
    \item [Chapter \ref{chapter:applications}] { Concludes the thesis by
        applying the improved XPFC model to the problem of multistep nucleation
        of nanoparticles and discusses potential future applications.
    } 
\end{description}
%

