In this chapter we'll describe the fundamental physics behind the phase field crystal theory.
Like many physical theories, it is derived using a set of successive approximations.
Each approximation yield a new theory that is more narrow in scope, but more tractible to either analytic or numerical analysis.

PFC is ultimately a thermodynamic theory and as such it makes connection to fundamental, microscopic physics by way of statistical mechanics.
Statistical mechanics tied the macroscopic observables to microscopic phenomena with a probabilistic approach.
The basic assumption at this level is that, if a system is sufficiently complex, there are circumstances under which its statistical behaviour becomes relatively simple.
Now for systems with macroscopic ($\mathcal{O}(10^{23})$) amount of particles this is almost always the case and so for this large systems we can often compute the macroscopic or thermodynamic observables using a statistical approach instead of solving the microscopic equations of motion.

At the fundamental level, our systems of interest are governed by quantum mechanics and so we might use the theory of quantum statistical mechanics to attempt to compute the thermodynamic observables of our system.
We will see that for our systems of interest that the quantum statistical theory is quite intractible, but with an approximation we can treat our system in the \textit{semi-classical limit}.

In semi-classical limit, we can talk in a robust way about the structure of the denisty field, using the Classical Density Functional Theory (CDFT) framework.
While CDFT supplies the correct setting to discuss the density, it is again rarely feasible to perform exact calculations.

Finally, we'll see that an approximation of the exact CFDT free energy functional will yield the PFC theory that is amenable to both analytic and numerical analysis.

\section{Statistical Mechanics in the Semi-classical limit}

Although the quantum statistical mechanics picture gives us a link between the microscopic and macroscopic reality of thermodynamics systems, it still contains too much detail for many systems of interest.
For instance, for many systems of interest, the precise bosonic or fermionic nature of the particles in the system has little consequence on the thermodynamic properties.
We can ignore some of these quantum mechanical details by looking at statistical mechanics in the \textit{semi-classical limit}.

For the sake of clarity, we'll look at a system of $N$ identical particles in the canonical ensemble but generalization to multi-component systems and other ensembles is straight forward. 
We start with the definition of the partition function for a system of many particles,  
\begin{equation}
    Z = Tr(e^{\beta \hat{H}}),
\end{equation} 
where, $\hat{H} = \frac{\vert\mathbf{p}\vert^2}{2m} + V(\mathbf{q})$ and $\mathbf{p} = (p_1, p_2, ...p_N)$ is the vector of particle momenta. 
$\mathbf{q}$ is similarly defined for the particle positions.
Wigner \cite{PhysRev.40.749},and, shortly after, Kirkwood \cite{PhysRev.44.31} showed that the partition function could be expanded in powers of $\hbar$, facilitating the calculation of both a classical limit and quantum corrections to the partition function.
Their method, the Wigner-Kirkwood expansion, involves evaluating the trace operation over a basis of plain wave solutions,
\begin{equation}
	\mathcal{Z}(\beta) = \int 
		\frac{\mathrm{d}\mathbf{q} \mathrm{d}\mathbf{p}}{(2\pi \hbar)^N}
		e^{-\frac{i\mathbf{p}\cdot\mathbf{q}}{\hbar}}
		e^{-\beta \hat{H}}
		e^{\frac{i\mathbf{p}\cdot\mathbf{q}}{\hbar}} = \int d\Gamma I(p, q),
\end{equation}
Where, $d\Gamma$ is the phase space measure $d\mathbf{p}d\mathbf{q}/(2\pi\hbar)^N$.
To compute the integrand, $I(p, q)$, we follow Uhlenbeck and Bethe \cite{Uhlenbeck1936729} and first compute its derivative,
\begin{equation}
	\frac{\partial I(p, q)}{\partial \beta} = -\pwp\hat{H}\pwm I(p, q).
\end{equation}
If we then make a change of variables, $I(q, p) = e^{-\beta H}W(q, p)$ use the explicit form of the Hamiltonian we arrive at a partial differential equation for $W$.
\begin{equation}
	\frac{\partial W}{\partial \beta} = \frac{\hbar^2}{2} \left(
		\nabla_{\mathbf{q}}^2 - 
		\beta(\nabla_{\mathbf{q}}^2V) + 
		\beta^2(\nabla V)^2 -
		2\beta(\nabla_{\mathbf{q}} V)\cdot\nabla_{\mathbf{q}} + 
		2 \frac{i}{\hbar}\mathbf{p}\cdot(\nabla_{\mathbf{q}} - \beta\nabla_{\mathbf{q}})
	\right)W(q, p)
\end{equation}
The solution can be written as a power series in $\hbar$, $W = 1 + \hbar W_1 + \hbar W_2 + ...$.
This creates a power series expansion for the partition function as well,
\begin{equation}
    \mathcal{Z} = (1 + \hbar \langle W_1 \rangle + \hbar^2 \langle W_2\rangle + ...) 
    \int d\Gamma e^{\beta\mathcal{H}}.
\end{equation}
Where the average, $\langle \cdot \rangle$, denotes the the classical average, 
\begin{equation}
    \langle A(p, q) \rangle = \frac{1}{\mathcal{Z}} 
        \int d\Gamma A(p, q) e^{-\beta \mathcal{H}}.
\end{equation}
For the sake of brevity we'll simply quote solution to second order, but details can be found in Landau and Lifshitz \cite{LANDAU198079}.
Interestingly, the first order term is exactly zero.
\begin{gather}
    \langle W_1 \rangle = 0 \\
    \langle W_2 \rangle = - \f{\beta^3}{24 m} \l\langle \left\vert\nabla_{\mathbf{q}}V\right\vert^2 \r\rangle
\end{gather}
In terms of the free energy, for example, the corrections to second order would be, 
\begin{equation}{\label{quantumcorr}}
    \mathcal{F} = \mathcal{F}_{classical} + \f{\hbar^2\beta^2}{24m}
        \l\langle \l\vert \nabla_{\mathbf{q}} V(\mathbf{q}) \r\vert^2 \r\rangle.
\end{equation}

There are a few things to note about this finding, first of all the correction inversely proportional to both the temperature and the particle mass.
For copper at room termperature, for instance, the prefactor $\hbar^2\beta^2/(24 m)$ is $\mathcal{O}(10^{-4})$. 
The correction is also proportional to the mean of the squared force felt by each particle. So high density materials will have a higher quantum correction because they sample the short-range repulsive region of the pair potential more than their low density counter parts.

\subsection{Indistinguishability}

There is an important distinction to be made between the quantum theory and the theory in the semi-classical limit.
The integral over phase space of the partition function must only take into account the \textit{physically different} states of the system.  
In the quantum theory this is acheived by tracing over any orthonormal basis of the Hilbert space, but in the classical theory we need to be careful not to double count states when identical particles are in the theory.
Exchange of two identical particles does not result in a physically different state and thus this state should only be considered only once in the sum over states in the partition function.
More precisely, we should write the classical partition function as,

\begin{equation}
    \mathcal{Z} = \int^\prime d\Gamma e^{-\beta \mathcal{H}},
\end{equation}
Where the primed integral denotes integration only over the physically distinct states. In the common case of $N$ identical particles, the phase space integral becomes, 
\begin{equation}
    \int^\prime d\Gamma \rightarrow \f{1}{N!}\int d\Gamma
\end{equation}

Aggregating our results, we can write the partition function in the semi-classical limit as,
\begin{equation}
    \mathcal{Z}(\beta) = \frac{1}{N!}\int d\Gamma e^{-\beta \mathcal{H}},
\end{equation}
Or, in the grand canonical ensemble,
\begin{equation}
    \Xi(\mu, \beta) = \sum_{N = 0}^\infty \f{e^{\beta\mu N}}{N!} \int d\Gamma e^{-\beta \mathcal{H}}
\end{equation}

Of course, this is exactly the form taught in introductory courses on statistical mechanics and derived by Gibbs\footnote{The $\hbar$ in Gibbs' formula was justified on dimensional grounds and was simply a scaling factor with units of action ($J\cdot s$)}prior to any knowledge of quantum mechanics \cite{Gibbs}.
The key insight here is to understand in a contolled way when this approximation is accurate and the magnitude of the next quantum correction is as seen in equation \ref{quantumcorr}.

\section{Classical Density Functional Theory}


