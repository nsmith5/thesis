In this chapter we will walk through three simplified binary PFC models. The
first is the original binary PFC model, which, while highly successful at
modelling a few important phenomena is ultimately limited in scope. The second
is the binary structural phase field crystal, or binary XPFC which was
successful in modelling a broad spectrum of crystalline structures, but was
limited in its ability of model liquid instablilities and a variety of phase
diagrams. Finally, we'll see a new contribution to which we will call the
regular phase field crystal model which is successful in modeling a broad
spectrum of invariant binary reactions and crystalline structures. 

All binary PFC models begin with a multicomponent variant of the approximate
free energy functional established in Chapter \ref{fundamentals},
%
\begin{align}
    \label{binary_cdft_free_energy}
    \beta\F[\A, \B] &= \sum_{i=A, B} \int \mathrm{d}r 
        \,\rho_i(r) \ln\l(\f{\rho_i(r)}{\rho_i^0}\r) - (1 - \beta\mu_i^0)\Delta\rho_i(r)\\
    &- \f{1}{2} \sum_{i,j=A, B} \Delta\rho_i(r) \ast C^{(2)}_{ij}(r, r^\prime) 
        \ast \Delta\rho_j(r^\prime). \nonumber
\end{align}
%
It is convenient to change variables to a dimensionless total density, $n(r)$ and local
concentration, $c(r)$,
%
\begin{gather}
    n(r) = \f{\Delta \rho}{\rho_0} = \f{\Delta\A + \Delta\B}{\A^0 + \B^0} \\
    c(r) = \f{\B}{\rho} = \f{\B}{\A + \B}.
\end{gather}
%
Scaling out a factor of the total reference density, $\rho_0$ we can break the free energy
functional in these new variables into two parts,
%
\begin{equation}
    \f{\beta\F[n, c]}{\rho_0} = \f{\beta\F_{id}[n, c]}{\rho_0} + \f{\beta\F_{ex}[n, c]}{\rho_0},
\end{equation}
%
Where,
%
\begin{align}
    \label{binary_ideal}
    \f{\beta\F_{id}[n, c]}{\rho_0} 
        &= \int \mathrm{d}r \,\l\lbrace (n(r) + 1)\ln(n(r) + 1) - (1 - \beta\mu^0)n(r) \r\rbrace \\
        &+ \int \mathrm{d}r \,\l\lbrace (n(r) + 1)\l(c\ln\l(\f{c}{c_0}\r) + (1-c)\ln\l(\f{1-c}{1-c_0}\r) \r)\r\rbrace, 
\end{align}
%
And, if we assume the local concentration $c(r)$ varies over much longer length
scales than the local density $n(r)$,
%
\begin{align}
    \label{binary_excess}
    \f{\beta\F_{ex}[n, c]}{\rho_0}
        = &-\f{1}{2} n(r) \ast \l[ 
            C_{nn}(r, r^\prime) \ast n(r^\prime) + C_{nc}(r, r^\prime) \ast \Delta c(r^\prime)\r] \\
        &-\f{1}{2} \Delta c(r) \ast \l[
            C_{cn}(r, r^\prime) \ast n(r^\prime) + C_{cc}(r, r^\prime) \ast \Delta c(r^\prime)\r]. \nonumber
\end{align}
%
We have introduced $\mu_0$ as the total chemical potential of the reference mixture, $c_0 = \B^0 / \rho_0$ as the
reference concentration and $\Delta c(r) = c(r) - c_0$ as the deviation of the concentration from the reference.
The $n-c$ pair correlation introduced in the excess free energy are,
%
\begin{align}
    C_{nn} &= \rho_0\l(c^2 C_{BB} + (1 - c)^2C_{AA} + 2c(1-c)C_{AB}\r) \\
    C_{nc} &= \rho_0\l(c C_{BB} - (1 - c) C_{AA} + (1 - 2c)C_{AB}\r) \\
    C_{cn} &= C_{nc} \\
    C_{cc} &= \rho_0\l(C_{BB} + C_{AA} - 2 C_{AB}\r)
\end{align}
%
Explicit calculations can be found in Appendix \ref{binary_correlations}. Differences in the various
binary PFC theories stem from different approximations of ideal free energy in equation \ref{binary_ideal} and
excess free energy in equation \ref{binary_excess}.

\section{Original Binary Phase Field Crystal Model}

\section{Binary Structural Phase Field Crystal Model}

\section{Regular Phase Field Crystal Model}
\subsection{Equilibrium Properties}
