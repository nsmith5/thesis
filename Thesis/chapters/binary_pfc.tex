In this chapter we will walk through three simplified binary PFC models. The
first is the original binary PFC model, which, while highly successful at
modelling a few important phenomena is ultimately limited in scope. The second
is the binary structural phase field crystal, or binary XPFC which was
successful in modelling a broad spectrum of crystalline structures, but was
limited in its ability of model liquid instablilities and a variety of phase
diagrams. Finally, we'll see a new contribution to which we will call the
regular phase field crystal model which is successful in modeling a broad
spectrum of invariant binary reactions and crystalline structures. 

All binary PFC models begin with a multicomponent variant of the approximate
free energy functional established in Chapter \ref{fundamentals},
%
\begin{align}
    \label{binary_cdft_free_energy}
    \beta\F[\A, \B] &= \sum_{i=A, B} \int \mathrm{d}r 
        \,\rho_i(r) \ln\l(\f{\rho_i(r)}{\rho_i^0}\r) 
        - (1 - \beta\mu_i^0)\Delta\rho_i(r)\\
    &- \f{1}{2} \sum_{i,j=A, B} \Delta\rho_i(r) \ast C^{(2)}_{ij}(r, r^\prime) 
        \ast \Delta\rho_j(r^\prime). \nonumber
\end{align}
%
It is convenient to change variables to a dimensionless total density, $n(r)$
and local concentration, $c(r)$,
%
\begin{gather}
    n(r) = \f{\Delta \rho}{\rho_0} = \f{\Delta\A + \Delta\B}{\A^0 + \B^0} \\
    c(r) = \f{\B}{\rho} = \f{\B}{\A + \B}.
\end{gather}
%
Scaling out a factor of the total reference density, $\rho_0$ we can break the
free energy functional in these new variables into three parts,
%
\begin{equation}
    \label{binary_total_free_energy}
    \f{\beta\F[n, c]}{\rho_0} = \f{\beta\F_{id}[n]}{\rho_0} 
        + \f{\beta\F_{mix}[n, c]}{\rho_0}
        + \f{\beta\F_{ex}[n, c]}{\rho_0},
\end{equation}
%
Where,
%
\begin{gather}
    \label{binary_ideal}
    \f{\beta\F_{id}[n]}{\rho_0} =
        \int \mathrm{d}r \,\l\lbrace (n(r) + 1)\ln(n(r) + 1) 
        - (1 - \beta\mu^0)n(r) \r\rbrace \\
    \label{binary_mixing}
    \f{\beta\F_{mix}[n, c]}{\rho_0} =
        \int \mathrm{d}r \,\l\lbrace (n(r) + 1)\l( 
            c\ln\l(\f{c}{c_0}\r) + (1-c)\ln\l(\f{1-c}{1-c_0}\r) \r)\r\rbrace, 
\end{gather}
%
And, if we assume the local concentration $c(r)$ varies over much longer length
scales than the local density $n(r)$,
%
\begin{align}
    \label{binary_excess}
    \f{\beta\F_{ex}[n, c]}{\rho_0}
        = &-\f{1}{2} n(r) \ast \l[ 
            C_{nn}(r, r^\prime) \ast n(r^\prime) 
          + C_{nc}(r, r^\prime) \ast \Delta c(r^\prime)\r] \\
        &-\f{1}{2} \Delta c(r) \ast \l[
            C_{cn}(r, r^\prime) \ast n(r^\prime) 
          + C_{cc}(r, r^\prime) \ast \Delta c(r^\prime)\r]. \nonumber
\end{align}
%
We have introduced $\mu_0$ as the total chemical potential of the reference
mixture, $c_0 = \B^0 / \rho_0$ as the reference concentration and $\Delta c(r)
= c(r) - c_0$ as the deviation of the concentration from the reference.  The
$n-c$ pair correlation introduced in the excess free energy are,
%
\begin{align}
    C_{nn} &= \rho_0\l(c^2 C_{BB} + (1 - c)^2C_{AA} + 2c(1-c)C_{AB}\r) \\
    C_{nc} &= \rho_0\l(c C_{BB} - (1 - c) C_{AA} + (1 - 2c)C_{AB}\r) \\
    C_{cn} &= C_{nc} \\
    C_{cc} &= \rho_0\l(C_{BB} + C_{AA} - 2 C_{AB}\r)
\end{align}
%
Explicit calculations can be found in Appendix \ref{binary_correlations}.
Differences in the various simplified binary PFC theories stem from differing
approximations of the terms in the free energy stated in equation
\ref{binary_total_free_energy}.

%%%%%%%%%%%%%%%%%%%%%%%%%%%%%%%%%%%%%%%%%%%%%%%%%%%%%
\section{Original Binary Phase Field Crystal Model} %
%%%%%%%%%%%%%%%%%%%%%%%%%%%%%%%%%%%%%%%%%%%%%%%%%%%%%

In the original simplified binary PFC theory, all terms in the free energy are
expanded about $n(r) = 0$ and $c(r) = c_0$ (ie., about their reference states).
For the ideal free energy this results in a polynomial truncated to fourth
order,
%
\begin{equation}
    \f{\beta\F_{id}[n]}{\rho_0} = \integrate{r}
    \l\lbrace \f{n(r)^2}{2} - \f{n(r)^3}{6} + \f{n(r)^4}{12} \r\rbrace.
\end{equation}
%
The linear term is dropped due to invariance in the equations of motion. If we
assume for simplicity of demonstration $c0 = 1/2$, the free energy of mixing
becomes a simple fourth order polynomial as well,
%
\begin{equation}
    \f{\beta\F_{mix}[n, c]}{\rho_0} = \integrate{r} \l\lbrace
       2\Delta c(r)^2 + \f{4\Delta c(r)^4}{3}
    \r\rbrace.
\end{equation}
%
Linear couplings to $n(r)$ are dropped by assuming, as we already have, that
the concentration field varies on a much longer length scale than the total
density and noting that the total density is defined about its average. This
argument can also be applied the linear couplings to $n(r)$ in the excess free
energy term which leaves only the $C_{nn}$ and $C_{cc}$ terms. Finally, these
two terms are approximated with a gradient expansions of the correlation
functions,
%
\begin{gather}
    C_{nn}(r, r^\prime) = \d(r - r^\prime)\l(
        \alpha + \beta \nabla^2 + \gamma \nabla^4 + \dots\r), \\
    C_{cc}(r, r^\prime) = \d(r - r^\prime)\l(
        \epsilon + \xi \nabla^2 + \dots\r).
\end{gather}
%
The expansion parameters, $\alpha, \beta,$ and $\gamma$ are all dependent on
temperature and concentration. We are required to expand $C_{nn}$ to fourth
order because, as noted in chapter \ref{dft_of_freezing} the peak of the direct
correlation function in Fourier space is the driving force for solidification.
The concentration field is correlated over a longer length scale implying that
only the short wavevectors are important in $C_{cc}$ so we can expand just to
quadratic order.

Gathering terms the resulting free energy functional for the original simplified
binary PFC model\footnote{The orignal simplified binary PFC model was
expressed using slightly different variables. We expand in $\Delta c(r)$ here to 
facilitate comparison with other theories} is,
%
\begin{align}
    \f{\beta\F[n, c]}{\rho_0} &= \integrate{r} \l\lbrace 
        \f{1}{2} n(r) \l( 1 - \alpha - \beta\nabla^2 - \eta\nabla^4 \r) n(r)
      - \f{n(r)^3}{6} + \f{n(r)^4}{12} \r\rbrace \\
    &+ \integrate{r} \l\lbrace
        \f{1}{2} \Delta c(r) \l( 4 - \epsilon - \xi\nabla^2 \r) \Delta c(r) 
      + \f{4 \Delta c(r)^4}{3} \r\rbrace. \nonumber
\end{align}
%

The strength of the original simplified binary PFC model is that is retains
most of the important physics of binary alloys in a very reduced theory. For
instance, the simplified model is capable of describing the equilibrium phase
diagrams of both eutectic alloys alloys and materials with a solid state
spinodal / liquid minimum.  Supplied with a diffusive equation of motion the
simplified model can model an impressive diversity of dynamic phenomena
including eutectic growth, phase segregation, dendritic growth, dislocation
motion in solid state spinodal coarsening and epitaxial growth.

The major limitation of the original simplified model is that the gradient
expansion of the 

{   
    \color{ForestGreen} List all of the applications which the original model,
    in its various forms, was successfully applied to. List failures if
    possible (here I'm specifically thinking about the Karma fitting problem
    for iron) and conceptual problems about modelling differect structures. 

    Additionally note the limitations of expansions in the 'concentration' when
    attempting to reproduce realistic phase diagram. Essentially, the divergent
    behaviour at the concentration boundaries makes this behaviour work.
}



%%%%%%%%%%%%%%%%%%%%%%%%%%%%%%%%%%%%%%%%%%%%%%%%%%%%%%%
\section{Binary Structural Phase Field Crystal Model} %
%%%%%%%%%%%%%%%%%%%%%%%%%%%%%%%%%%%%%%%%%%%%%%%%%%%%%%%

{
    \color{ForestGreen} Derive the Binary XPFC free energy and note differences
    from the derivation used for the original simplified model. List all of the
    novel applications that where approached using this model. List the theoretical
    advantages (better phase diagrams, precise control of structure etc)

    List disadvantages (liquid is ideal, narrow spectrum of phase diagrams can
    be reproduces). List theoretic disadvantages (ideal mixing instead of more
    realistic regular mixing basically).
}

%%%%%%%%%%%%%%%%%%%%%%%%%%%%%%%%%%%%%%%%%%%%%
\section{Regular Phase Field Crystal Model} %
%%%%%%%%%%%%%%%%%%%%%%%%%%%%%%%%%%%%%%%%%%%%%

{

}

%%%%%%%%%%%%%%%%%%%%%%%%%%%%%%%%%%%%%
\subsection{Equilibrium Properties} %
%%%%%%%%%%%%%%%%%%%%%%%%%%%%%%%%%%%%%
