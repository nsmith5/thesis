In this chapter we will walk through three simplified binary PFC models. The
first is the original binary PFC model, which, while highly successful at
modelling a few important phenomena is ultimately limited in scope. The second
is the binary structural phase field crystal, or binary XPFC which was
successful in modelling a broad spectrum of crystalline structures, but was
limited in its ability of model liquid instablilities and a variety of phase
diagrams. Finally, we'll see a new contribution to which we will call the
regular phase field crystal model which is successful in modeling a broad
spectrum of invariant binary reactions and crystalline structures. 

\section{Original Binary Phase Field Crystal Model}

To derive the original binary PFC model, we begin with a multicomponent variant of
equation \ref{cdft_free_energy}, 
%
\begin{align}
    \beta\F[\A, \B] &= \sum_{i=A, B} \int \mathrm{d}r \rho_i \ln\l(\f{\rho_i}{\rho_i^0}\r) 
            - (1 - \mu_i^0) \Delta\rho_i \\
        &- \f{1}{2} \sum_{i,j=A, B} \int \mathrm{d}r \mathrm{d}r^\prime \Delta\rho_i(r) 
            C^{(2)}_{ij}(r, r^\prime) \Delta\rho_j(r^\prime).
\end{align}
%
As noted in chapter \ref{dft_of_freezing}, we can produce a PFC model by approximating
the ideal component of the free energy. 

\section{Binary Structural Phase Field Crystal Model}

\section{Regular Phase Field Crystal Model}

\section*{Derivation of the General Binary XPFC Free Energy}

Now that we've seen the recipe for building a phase-field crystal theory for a
single component system the process for building a multicomponent theory
proceeds analogously with a few minor changes in perspective.  We start again
by looking at the ideal and excess contributions to the free energy.

\subsubsection{Ideal Free Energy to Two Components} The kinetic energy terms of
each species in the Hamiltonian give rise to seperate contributions to the free
energy as you might expect.  We'll label the two species A and B.

\begin{equation} \beta\F^{tot}_{id}[\rho_A, \rho_B] = \beta\F_{id}[\rho_A] +
\beta\F_{id}[\rho_B] \end{equation} Where,
\begin{description}[labelindent=10pt, labelsep=10pt] \item[$\beta\F_{id}$] is
            the same ideal free energy functional as previously
\end{description}

\subsubsection{Excess Free Energy of a Two Component System} Our expansion of
the excess free energy works just as before but we must sum over the
contributions from each species.

\begin{equation} \beta\F_{ex} = \beta\F_{ex}^0 - \int \,dr C^{(1)}_{i}(r)
\Delta\rho_i(r) - \f{1}{2} \int dr \int dr^\prime \Delta\rho_i(r)
C^{(2)}_{ij}(r, r^\prime) \Delta\rho_j(r^\prime) \end{equation}

Where indices denote species (A or B) and repeated indices are summed over.

\subsubsection{Total free energy of a Two Component System}

Putting together the excess and ideal terms together and dropping the constant
and linear terms as we did previously we find the following total free energy,

\begin{equation} \beta\F[\rho_A, \rho_B] = \int dr \l\lbrace \Delta\rho_i
\ln\l(\f{\Delta\rho_i}{\rho_{i0}}\r) - \Delta\rho_i\r\rbrace - \f{1}{2} \int dr
\int dr^\prime \Delta\rho_i(r) C^{(2)}_{ij}(r, r^\prime) \Delta
\rho_j(r^\prime) \end{equation}

\subsubsection{Changing variables} Typically, the concentration is the variable
we care about in binary systems so instead of preceeding with the usual phase
field crystal approximations at this point we make a change of variables to
concentration, $c$, and total density, $\rho$.

\begin{align} \rho &= \rho_A + \rho_B             & \rho_0 &= \rho_{0A} +
\rho_{0B} \nonumber \\ c &= \f{\rho_A}{\rho_A + \rho_B}    &  c_0 &=
\f{\rho_{0A}}{\rho_{0A} + \rho_{0B}} \nonumber \end{align}

Making this change of variables and seperating total density and concentration
contribution we find a free energy functional of the form,

\begin{equation} \beta\F[c, \rho] = \end{equation}
