\label{chapter:cdft_intro}

Many physical theories are derived using a succession of approximations. While
each approximation yields a theory that is more narrow in scope, it is
typically more tractible to either analytical or numerical analysis.  Classical
Density Functional Theory (CDFT) is derived using this approach and in this
chapter we'll examine each approximation and the intermediate theory they
supply. 

CDFT is a theory of statistical mechanics. This means CDFT connects microscopic
physics to macroscopic observables using statistical
inference\footnote{Statistical mechanics is not always described as statistical
inference. See works of E. T. Jaynes for details on this approach
\cite{JAYNES57}} instead of attempting to compute microscopic equations of
motion. The microscopic physics in this case is most accurately described by
many-body quantum mechanics and so the theory of quantum statistical mechanics
is a natural starting point in any attempt to calculate thermodynamic
observables.

We will see that for our systems of interest that the full quantum statistical
theory is completely intractible. To preceed, we'll look at quantum statistical
mechanics in the \textit{semi-classical limit}. In the semi-classical limit
we'll develop a theory of inhomogenous fluids called Classical Density
Functional Theory (CDFT). Finally, we'll see that constructing exact free
energy functionals for CDFT is rarely possible and look at an approximation
scheme for these functionals.

%%%%%%%%%%%%%%%%%%%%%%%%%%%%%%%%%%%%%%%%%%%%%%%%%%%%%%%%%%%%%
\section{Statistical Mechanics in the Semi-classical limit} %
%%%%%%%%%%%%%%%%%%%%%%%%%%%%%%%%%%%%%%%%%%%%%%%%%%%%%%%%%%%%%

At a microscopic level, all systems are governed by the fundamental physics of
quantum mechanics. Statistical mechanics and in particular quantum statistical
mechanics provides a map between this microscopic reality and macroscopic
thermodynamic observables. For most applications, quantum statistical mechanics
is both intractable to analysis and contains more detail than necessary. For
instance, the precise bosonic or fermionic nature of the particles in the
system often has little consequence on the thermodynamic properties.  We can
ignore some of these quantum mechanical details by looking at statistical
mechanics in the \textit{semi-classical limit}.

For the sake of clarity, we'll look at a system of $N$ identical particles in
the canonical ensemble which is straightforward to generalize to
multi-component systems and other ensembles. We start with the definition of
the partition function for a system of many particles,  
%
\begin{equation}
    \label{eq:quantum_partition}
    Z = \trace{e^{-\beta\hat{H}}}, 
\end{equation} 
%
where, 
%
\begin{description}[align=right, labelwidth=1cm]
    \item[$\hat{H}$]{ is the Hamiltonian $\f{\vert\hat{\p}\vert^2}{2m} +
        V(\hat{\q})$, 
    }
    \item[$\p$] is set of particle momenta $(p_1, p_2, ...p_N)$,
    \item[$\q$] is similarly the set of particles positions, and,
    \item[$\beta$]{ is the inverse temperature $1 / k_b T$ where $k_b$ is the
        Boltzmann constant.
    }  
\end{description}
%
Wigner \cite{PhysRev.40.749}, and shortly after, Kirkwood \cite{PhysRev.44.31}
showed that the partition function could be expanded in powers of $\hbar$,
facilitating the calculation of both a classical limit and quantum corrections
to the partition function.  Their method, the Wigner-Kirkwood expansion,
involves evaluating the trace operation over a basis of plane wave solutions,
%
\begin{equation}
    \Z(\beta) = \int \f{\mathrm{d}\q \mathrm{d}\p}{(2\pi \hbar)^N}
        e^{-\frac{i\p\cdot\q}{\hbar}} 
        e^{-\beta \hat{H}} e^{\frac{i\p\cdot\q}{\hbar}} 
        = \int d\Gamma I(\q, \p), 
        \label{plane_wave_trace}
\end{equation}
%
Where, $\mathrm{d}\Gamma$ is the phase space measure
$\mathrm{d}\p\mathrm{d}\q/(2\pi\hbar)^N$.  To compute the integrand, $I(\q,
\p)$, we follow Uhlenbeck and Bethe \cite{Uhlenbeck1936729} and first compute
its derivative,
%
\begin{equation}
    \frac{\partial I(\q, \p)}{\partial \beta} = -\pwp\hat{H}\pwm I(\q, \p).
     \label{I_deriv}
\end{equation}
%
We then make a change of variables, $I(\q, \p) = e^{-\beta\ham}W(\q, \p)$,
where $\ham$ is the {\it classical} Hamiltonian. The new function $W(\q, \p)$ encodes
the deviation from classical behaviour due to a lack of commutation of the
potential and kinetic energy terms in the Hamiltonian. Substituting this 
redefined form of $I(\q, \p)$ into equation \ref{I_deriv}, using the explicit
form of the quantum Hamiltonian and after a considerable amount of algebra we
find a partial differential equation for $W$,
%
\begin{equation} 
    \label{eq:uhlenbeck} 
    \f{\partial W}{\partial \beta} = \f{\hbar^2}{2} \l(
        \nabla_{\mathbf{q}}^2 
        - \beta(\nabla_\q^2V)
        + \beta^2(\nabla V)^2 
        - 2\beta(\nabla_\q V)\cdot\nabla_\q
        + 2 \frac{i}{\hbar}\p\cdot(\nabla_\q -\beta\nabla_\q) 
    \r)W(\q, \p).
\end{equation}
%
As in typical in perturbation theories, the solution can be expanded in a power
series of a small number, in this case, $\hbar$, according to $W = 1 + \hbar
W_1 + \hbar^2 W_2 + ...$. By substituting  this expansion into $I(\q, \p) =
e^{-\beta\ham}W(\q, \p)$  and $I(\p,\q)$  back into equation
\ref{plane_wave_trace} we find a power series expansion for the partition
function as well,
%
\begin{equation}
    \label{eq:semi-classical-partition}
    \mathcal{Z} = \left(1 + \hbar \mean{W_1} + \hbar^2 \mean{W_2}+ ...\right)
        \int d\Gamma e^{\beta\ham}.
\end{equation}
%
Where the average, $\mean{\cdot}$, denotes the the classical average, 
%
\begin{equation} \langle A(p, q) \rangle = \frac{1}{\Z} \int d\Gamma A(p, q)
e^{-\beta \ham}.  \end{equation}
%
Solving equation \ref{eq:uhlenbeck} to second order in $\hbar$ and computing
the classical averages in equation \ref{eq:semi-classical-partition} the
quantum corrections to the classical partition are computed to second 
order as\footnote{For detailed calculations see \cite{LANDAU198079}.},
%
\begin{gather}
    \mean{W_1}= 0, \\
    \mean{W_2} = - \f{\beta^3}{24 m} \mean{\l\vert\nabla_\q V \r\vert^2}.
\end{gather}
%
The first order term is zero because $W_1(\q, \p)$ is an odd function of $\p$.
In terms of the Helmholtz free energy, for example, the corrections to second
order would be, 
%
\begin{equation}
    \label{quantumcorr}
    \mathcal{F} = \mathcal{F}_{classical} + \f{\hbar^2\beta^2}{24m}
        \mean{ \l\vert \nabla_\q V(\q) \r\vert^2 }.
\end{equation}

There are a few items of importance in equation \ref{quantumcorr}. First of
all, the correction is inversely proportional to both the temperature and the
particle mass.  For copper at room temperature, for instance, the prefactor
$\hbar^2\beta^2/(24 m)$ is $\mathcal{O}(10^{-4})$ or at its melting temperature
the prefactor is $\mathcal{O}(10^{-6})$.  The correction is also proportional
to the mean of the squared force felt by each particle. So high density
materials will have a higher quantum correction because they sample the
short-range repulsive region of the pair potential more than their low density
counter parts.

%%%%%%%%%%%%%%%%%%%%%%%%%%%%%%%%%%%
\subsection{Indistinguishability} %
%%%%%%%%%%%%%%%%%%%%%%%%%%%%%%%%%%%

There is an important distinction to be made between the quantum theory and the
theory in the semi-classical limit.  The integral over phase space of the
partition function must only take into account the \textit{physically
different} states of the system.  In the quantum theory this is achieved by
tracing over any orthonormal basis of the Hilbert space, but in the classical
theory we need to be careful not to double count states involving identical
particle configurations. Classically, exchange of two identical particles does
not result in a physically different state and thus these states should be
considered only once in the sum over states in the partition function.  More
precisely, we should write the classical partition function as,
%
\begin{equation}
    \Z = \int^\prime d\Gamma e^{-\beta \ham}, 
\end{equation}
%
Where the primed integral denotes integration only over the physically distinct
states. In the common case of $N$ identical particles, the phase space integral
becomes, 
%
\begin{equation}
    \int^\prime d\Gamma \rightarrow \f{1}{N!}\int d\Gamma
\end{equation}
%
Aggregating our results, we can thus write the partition function in the
semi-classical limit as,
%
\begin{equation}
    \Z(\beta) = \frac{1}{N!}\int d\Gamma e^{-\beta \ham} + \mathcal{O}(\hbar ^ 2),
\end{equation}
%
Or, in the grand canonical ensemble,
%
\begin{equation} 
    \Xi(\mu, \beta) = \sum_{N = 0}^\infty \f{e^{\beta\mu N}}{N!}
        \int d\Gamma \l( e^{-\beta \ham} + \mathcal{O}(\hbar^2)\r)
\end{equation}

Of course, to first order in $\hbar$, this is exactly the form taught in
introductory courses on statistical mechanics and derived by Gibbs\footnote{The
$\hbar$ in Gibbs' formula was justified on dimensional grounds and was simply 
introduced as a scaling factor with units of action ($J\cdot s$)} prior to any knowledge of
quantum mechanics \cite{Gibbs}. The key insight here is to understand, in a
controlled way, when this approximation is accurate and the magnitude of the next
quantum correction is as seen in equation \ref{quantumcorr}. We now apply this
semi-classical limit of statistical mechanics to the study of the local density
field.

%%%%%%%%%%%%%%%%%%%%%%%%%%%%%%%%%%%%%%%%%%%%%%%%
\section{Classical Density Functional Theory}  %
%%%%%%%%%%%%%%%%%%%%%%%%%%%%%%%%%%%%%%%%%%%%%%%%

Ostensibly, when we study formation and evolution of microstructure in solids,
our observable of interest is the density field.  As per usual in theories of
statistical thermodynamics we must distinguish between microscopic operators
and macroscopic observables (the later being the ensemble average of the
former).  In classical statistical mechanics, operators are simply functions
over the phase space, $\Gamma$.  We use the term operator to make connection
with the quantum mechanical theory.  In the case of the density field, the
microscopic operator is the sum of Dirac delta functions at the position of each
particle,
%
\begin{equation} 
    \hat{\rho}(x; \q) = \sum_{i = 0}^N \d^{(3)}\l(x - q_i\r)
\end{equation}
%
From which the thermodynamic observable is, 
%
\begin{equation} 
    \label{mean_density} 
    \rho(x) = \mean{\hat{\rho}(x; \q)} = 
        \trace{\hat{\rho}(x; \q) f(\q, \p)}
\end{equation}
%
Where, $\trace{\cdot}$ now denotes the classical trace,
%
\begin{equation}
    \trace{A(\q, \p) f(\q, \p)} \equiv \sum_{N =0}^\infty
        \f{1}{N!}\int\mathrm{d}\Gamma A(\q, \p) f(\q, \p), 
\end{equation}
%
And, $f(\q, \p)$ is the equilibrium probability density function,
%
\begin{equation} 
    f(\q, \p) = \f{e^{-\beta (\ham - \mu N)}}{\Xi(\mu, \beta)}.
\end{equation}
%
where $\ham$ is the classical Hamiltonian, $\mu$ the chemical potential of the
system and $\Xi(\mu,\beta)$ is the grand partition function of the system.

To construct a theory of the density field we review the usual methodology for
statistical thermodynamics. We will do so in the frame of entropy maximization
in which the entropy is maximized subject to the macroscopically available
information. Taking the existence of an average of the density field, particle
number and energy as the macroscopically available information, we can maximize
then Gibb's entropy functional,
%
\begin{equation}
    S[f(\q, \p)] = -k_b \trace{f(\q, \p)\ln\l(f(\q, \p)\r)}, 
\end{equation}
%
subject to the aforementioned constraints (fixed average density, particle
number and total energy) to find a probability density function of the form,
%
\begin{equation} 
    f(\q, \p) \propto \exp\l\lbrace-\beta \l(\ham -\mu N + \int \mathrm{d}x
        \phi(x)\hat{\rho}(x)\r)\r\rbrace.
\end{equation}
%
Where, $\beta$, $\mu$ and $\phi(x)$ are the Lagrange multipliers associated
with constraints of average energy, number of particles and density
respectively. As you might imagine, the constraints of average particle number
and density are not independent and satisfy,
%
\begin{equation}
    N = \int dx \hat{\rho}(x),
\end{equation}
%
We can combine their Lagrange multipliers into one,
%
\begin{equation}
    f(\q, \p) \propto \exp\l(- \beta(\ham - \int \mathrm{d}x \psi(x)
        \hat{\rho}(x))\r),
\end{equation}
%
Where, $\psi(x) = \mu - \phi(x)$, is the combined Lagrange multiplier named the
\textit{intrinsic chemical potential}. Recalling that chemical potential is the
change in Helmholtz free energy made by virtue of adding particles to the
system,
%
\begin{equation}
    \label{eq:chem_potential} 
    \f{\partial F}{\partial N} = \mu,
\end{equation}
%
The interpretation of the intrinsic chemical potential follows as the Helmholtz
free energy change due to particles being added to a specific location.  We'll
see this in more detail briefly where we'll see an analogous equation for the
intrinsic chemical potential.

The objective of statistical theories is to compute the statistics of some
observable (random variable) of choice. Two special sets of statistics provide
a complete description of the observable's probability distribution: the
\textit{moments} and \textit{cumulants}\footnote{See \cite{KUBO62} for
discussion of moments, cumulants and their importance in statistical
mechanics}.  The calculation of moments and cumulants can be aided by use of
generating functions. In the case of statistical mechanics the generating
functions of moments and cumulants have special physical significance. The
generating function of moments is closely related to the partition function and
the generating function of cumulants is closely related to the associated
thermodynamic potential. 

In the case where the observable is the local density field, this is made somewhat
more technical by the fact that the density is a function instead of a scalar
variable.  As such the partition function is more precisely called the
partition \textit{functional} as it depends on a function as input. The
thermodynamic potential will thus also be a functional.  Specifically, the grand
canonical partition functional is,
%
\begin{equation}
    \Xi[\psi(x)] = \trace{\exp\l(-\beta\ham +\beta\int\mathrm{d}x
        \psi(x) \hat{\rho}(x)\r)}.
\end{equation}
%
As alluded to above, the partition functional is a type of moment generating
functional in the sense that repeated (functional) differentiation with respect 
to  the intrinsic  chemical potential yields moments of the density field:
%
\begin{equation}
    \f{\beta^{-n}}{\Xi}\,\, \f{\d^n \Xi[\psi]}{\d \psi(x_1) \dots \d\psi(x_n)} 
        = \mean{\hat{\rho}(x_1)\dots\hat{\rho}(x_n)}.
\end{equation}
%
Similarly, we can construct a thermodynamic potential by taking the logarithm
of the partition function. This potential in particular is called the
\textit{grand potential functional} in analogy with the grand potential of
thermodynamics,
%
\begin{equation}
    \Omega[\psi(x)] = -k_bT\log\l(\Xi[\psi(r)]\r).
\end{equation}
%
The grand potential functional is a type of cumulant generating functional in
the sense that repeated functional differentiation yields cumulants of the
density field:
%
\begin{equation}
    -\beta^{-n + 1} \f{\d^n\Omega[\psi]}{\d\psi(x_1)\dots\d\psi(x_n)}
        = \mean{\hat{\rho}(x_1)\dots\hat{\rho}(x_n)}_c
\end{equation}
%
Where, $\langle \cdot \rangle_c$, denotes the cumulant average \cite{KUBO62}.

If we examine the first two cumulants,
%
\begin{gather}
    \label{mean_rho}
    - \f{\d \Omega[\psi]}{\d \psi(x)}
        = \mean{\hat{\rho}(x)} \equiv \rho(x), \\
    \label{var_rho} 
    -k_b T\f{\d^2 \Omega[\psi]}{\d \psi(x) \d \psi(x^\prime)}
        = \mean{(\hat{\rho}(x) - \rho(x))
          (\hat{\rho}(x^\prime) - \rho(x^\prime))}.
\end{gather}
%
We notice two remarkable things: The first, implies that the average density
field is a function of only its conjugate field, the intrinsic chemical
potential, and the second implies that that relationship is
invertible\footnote{The inverse function theorem only implies local
invertibility, there is no guarentee of global invertibility. Indeed phase
coexistance is a manifestation of this fact where a single intrinsic chemical
potential is shared by two phases}.  To see this, we compute the Jacobian by
combining equation \ref{mean_rho} and \ref{var_rho},
%
\begin{equation}
    \label{jacobian}
    \f{\d \rho(x)}{\d \psi(x^\prime)} 
        = \beta \mean{(\hat{\rho}(x) - \rho(x))
        (\hat{\rho}(x^\prime) - \rho(x^\prime))}.
\end{equation}
%
The right hand side of equation \ref{jacobian} is an autocorrelation function
and therefore positive semi-definite by the Weiner-Khinchin theorem
\cite{ESPANOL09}. This implies that, at least locally, the intrinsic chemical
potential can always be written as a functional of the average density,
$\psi[\rho(x)]$, and vice versa.  Furthermore, because all of the higher order
cumulants of the density depend on the intrinsic chemical potential, they too
depend only on the average density.

Given the importance of the average density, $\rho(x)$, it follows that we
would like to use a thermodynamic potential with a natural dependence on the
density.  We can construct a generalization of the Helmholtz free energy that
has precisely this characteristic by Legendre transforming the Grand potential,
%
\begin{equation}
    \F[\rho(x)] = \Omega[\psi[\rho]] + \int dx \rho(x) \psi(x).
    \label{intrinsic_F}
\end{equation}
%
$\F[\rho(x)]$ is called the \textit{intrinsic free energy functional}.

It can be shown \cite{HansenAppendixB} that $\rho(x)$ must be the global
minimum of the grand potential, which sets the stage for the methodology of
classical density functional theory: if we have a defined intrinsic free energy
functional, $\F$, we can find the equilibrium density field by solving the
asssociated Euler-Lagrange equation, 
%
\begin{equation}
    \f{\d \Omega[\rho]}{\d \rho(r)} = 0.
\label{fundamental_CDFT}
\end{equation}
%

Finally, we may construct an analogous equation to equation
\ref{eq:chem_potential} for the intrinsic chemical potential,
%
\begin{equation}
    \label{eq:int_chem_potential} 
    \f{\d \F}{\d \rho(x)} = \psi(x),
\end{equation}
%
which follows from equation \ref{intrinsic_F} assuming equation
\ref{fundamental_CDFT}. Equation \ref{eq:int_chem_potential} implies that the
intrinsic chemical potential is the free energy cost of adding density to the
location $x$ specifically. 

%%%%%%%%%%%%%%%%%%%%%%%%%%%%%%%%%%%%%%%%%%%%%%%%%%%
\section{Techniques in Density Functional Theory} %
%%%%%%%%%%%%%%%%%%%%%%%%%%%%%%%%%%%%%%%%%%%%%%%%%%%

The difficulty in formulating a density functional theory is the construction
of an appropriate free energy functional.  While exact calculations are rarely
feasible, there are a variety of techniques that help in building approximate
functionals.  It is important to note first what we \textit{can} compute
exactly.  In the case of the ideal gas, we can compute the grand potential and
free energy functional exactly,
%
\begin{gather}
    \Omega_{id}[\psi] = -\f{k_bT}{\Lambda^{3}} 
        \int\mathrm{d}x\,e^{\beta\psi(x)} \\ 
    \F_{id}[\rho] = k_bT\int \mathrm{d}x
        \l\lbrace \rho(x) \ln\l(\Lambda^3\rho(x)\r) - \rho(x) \r\rbrace,
    \label{F_ideal}
\end{gather}
% 
Where $\Lambda$ is the thermal de Broglie wavelength,
%
\begin{equation}
    \Lambda = \sqrt{\f{2 \pi \hbar^2}{m k_b T}}.
\end{equation}
%
We may then express deviation from ideality by factoring the ideal contribution
out of the partition function,
%
\begin{equation}
    \Xi[\psi] = \Xi_{id}[\psi]\Xi_{ex}[\psi],
\end{equation}
%
leading to grand potential and free energy functionals split into ideal and
\textit{excess} components,
%
\begin{gather}
    \Omega = \Omega_{id} + \Omega_{ex} \\
    \F = \F_{id} + \F_{ex}.
\end{gather}

The interaction potential, $V(\q)$, in the excess partition function typically
makes a direct approach to calculating the excess free energy intractible.
Though perturbative methods, including the cluster expansion technique
\cite{MAYER41}, have been developed to treat the interaction potential
systematically, other approximation schemes for the excess free energy are
typically more pragmatic, particularly where deriving models that are tractable
for the numerical simulation of dynamics is concerned.  In particular, we can
approximate the excess free energy by expanding around a reference homogeneos
fluid with chemical potential $\mu_0$ and density $\rho_0$,
%
\begin{equation}
    \label{fex_expansion}
    \F_{ex}[\rho] = \F_{ex}[\rho_0]
        + \l . \f{\d \F_{ex}}{\d\rho(x)} \r\vert_{\rho_0} \ast \Delta\rho(x) 
        + \f{1}{2} \Delta\rho(x^\prime) \ast
            \l . \f{\d^2 \F_{ex}}{\d\rho(x)\d\rho(x^\prime)}
            \r\vert_{\rho_0} \ast \Delta\rho(x) 
        + \dots,
\end{equation}
%
where $\Delta\rho(x) = \rho(x) - \rho_0$ and we have introduced the notation,
$\ast$ to mean integration over repeated co-ordinates, for example,
%
\begin{equation}
    f(x^\prime) \ast g(x^\prime)
        \equiv \int\mathrm{d}x^\prime f(x^\prime) g(x^\prime).
\end{equation}
%
The excess free energy is the generating functional of a family of correlation
functions called \textit{direct correlation functions}, 
%
\begin{equation}
    \label{direct_correlation}
    \f{\d^n \F_{ex}[\rho]}{\d \rho(x_1) ... \d \rho(x_n)}
        = -\beta C^{n}(x_1, \dots, x_n),
\end{equation}
%
the first of which, for a uniform fluid, is the excess contribution to the
chemical potential. We may express  this as the total chemical potential less
the ideal contribution (see equation \ref{F_ideal}), 
%
\begin{equation}
    \label{mu_ex}
    \left.\f{\d F_{ex}}{\d \rho}\right\vert_{\rho_0}
        = \mu_0^{ex}
        = \mu_0 - \mu_{id} 
        = \mu_0 - k_bT\ln\left(\Lambda^3 \rho_0\right).
\end{equation}
%
Truncating the expansion in equation \ref{fex_expansion} to second order in
$\Delta\rho(x)$ and substituting the linear and quadratic terms from  equation
\ref{mu_ex} and \ref{direct_correlation}, we can simplify the excess free
energy to,
%
\begin{equation}
    \label{F_ex}
    \F_{ex}[\rho(r)] = \F_{ex}[\rho_0] 
        + \int dr 
            \l\lbrace\mu - k_bT \ln\l(\Lambda^3 \rho_0 \r)
            \r\rbrace \Delta\rho(r)
        - \f{k_b T}{2} \Delta\rho(r) \ast C^{(2)}_0(r, r^\prime) 
            \ast \Delta\rho(r^\prime),
\end{equation}
%
where $C^{(2)}_0(r, r^\prime)$ denotes the two-point direct correlation
function at the reference state.  Combining equation \ref{F_ideal} with the
simplified excess free energy in equation \ref{F_ex}, we can express total
change in free energy, $\Delta \F = \F - \F[\rho_0]$, as,
%
\begin{equation}
    \label{cdft_free_energy}
    \Delta \F [\rho(r)] 
        = k_b T\int \mathrm{d}r 
            \l\lbrace \rho(r) \ln\l(\f{\rho(r)}{\rho_0} \r) 
            - (1-\beta\mu_0) \Delta \rho(r)\r\rbrace 
        - \f{k_b T}{2}\Delta \rho(r) \ast C_0^{(2)}(r, r^\prime) 
            \ast \Delta \rho(r^\prime).
\end{equation}
%
We find an equivalent expression for the grand potential after a Legendre
transform,
%
\begin{equation}
    \label{cdft_grand_potential}
    \Delta \Omega [\rho(r)]
        = k_b T\int \mathrm{d}r 
            \l\lbrace \rho(r) \l[\ln\l(\f{\rho(r)}{\rho_0} \r) 
            + \beta\phi(r) \r] -  \Delta \rho(r)\r\rbrace 
        - \f{k_b T}{2}\Delta \rho(r) \ast C_0^{(2)}(r, r^\prime) 
            \ast \Delta \rho(r^\prime),
\end{equation}
%
where $\phi(r)$ is defined as an external potential, introduced into the system
for completeness.

We see that the density functional theory derived here can be derived through a
series of approximations from a fundamental basis in quantum statistical
mechanics  and requires no more parameters than the thermodynamic details of a
homogeneous  reference fluid. It is reasonable to ask at this point whether or
not we have really gained anything with this approximation scheme.  Although we
have arrived at a relatively simple form for the free energy functional, we've
added several parameters to the functional based on the reference fluid.
Thankfully, the theory of homogeneous liquids is very well established.  This
implies we may rely on a broad choice of analytical, numerical or experimental
techniques to derive these parameters.

Equation \ref{cdft_free_energy} establishes an approximate density functional
theory for inhomogenous fluids. However, as we will see in the following
chapter, the properties of the direct correlation function  $C_o^2(r,r^\prime)$
also carries information  about how the fluid solidifies in the solid state as
temperature or density cross into the coexistence.
