\section*{Abstract}
\label{sec:abstract}
\addcontentsline{toc}{chapter}{\nameref{sec:abstract}}

Two improvements to the binary Structural Phase Field Crystal (XPFC) theory are
presented. The first is a general phenomenology for modelling density-density
correlation functions and the second extends the free energy of mixing term in
the binary XPFC model beyond ideal mixing to a regular solution model. These
improvements are applied to study kinetics of precipitation from solution. We
see a two-step nucleation pathway similar to recent experimental work
\cite{LOH17, WALLACE13} in which the solution first decomposes into solute-poor
and solute-rich regions followed by nucleation in the solute-rich regions.
Additionally, we find a phenomenon not previously described in literature in
which the growth of precipitates is accelerated in the presence of
uncrystallized solute-rich regions.

\clearpage

\section*{Abrégé}
\label{sec:abrege}
\addcontentsline{toc}{chapter}{\nameref{sec:abrege}}

Nous présentons deux ameliorations au theorie "Structural Phase Field Crystal"
(XPFC) binaire. Le premier decrit une phénoménologie pour une modèle des
fonctions de corrélations des densités, et le deuxième augmente la modèle XPFC
binaire au delà de la modèle idéale en ajutant une terme au énergie libre de
Helmholtz. Ces améliorations sont appliqués aux études kinétiques de la
précipité d'une solution. Nous voyons un chemin de nucléation similaire aux
éxpériments réçentes \cite{LOH17, WALLACE13} dans lequel le solution se sépare
en regions avec concentration de soluté bas et élèvé suivi par nucléation dans
les régions avec hautes concentrations de soluté. De plus, nous decouvrons une
acceleration de l'acroissement du precipité en presence des regions avec une
concentration de soluté élèvé. Ce dernier est une phénomène auparavant pas
décrit en literature. 
