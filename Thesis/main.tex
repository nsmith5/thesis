\documentclass[12pt,Bold,letterpaper]{mcgilletdclass}

%%%%%%%%%%%%
% Packages %
%%%%%%%%%%%%

% UTF8 encoding
\usepackage[utf8]{inputenc}

% Graphics and path to graphics
\usepackage[dvips, final]{graphicx}
\graphicspath{ {figures/} }

% Margins etc
\usepackage[dvips]{geometry}

% Math packages
\usepackage{amsmath}
\usepackage{array}
\usepackage{mathtools}
\usepackage{enumitem}

% Page alignment \addtolength{\hoffset}{0pt}
\addtolength{\voffset}{0pt}

%%%%%%%%%%%%%%%%
% Front matter %
%%%%%%%%%%%%%%%%

\SetTitle{\huge{Structureal Phase Field Crystal Models as a Lense on Nonequilibrium Kinetic Pathways}}
\SetAuthor{Nathan Frederick Smith}
\SetDegreeType{Masters of Science}
\SetDepartment{Department of Physics}
\SetUniversity{McGill Univsersity}
\SetUniversityAddr{Montreal, Quebec}
\SetThesisDate{2017-??-??}
\SetRequirements{[requirements statement]}
\SetCopyright{[copyright statement]}

%%%%%%%%%%%%
% Indicies %
%%%%%%%%%%%%

\makeindex[keylist]
\makeindex[abbr]

%%%%%%%%%%%%%%%%%%%%%%%%
% Aliases and commands %
%%%%%%%%%%%%%%%%%%%%%%%%

\renewcommand{\d}{\delta}       % Dirac delta
\newcommand{\F}{\mathcal{F}}    % Free energy Functional
\renewcommand{\l}{\left}        % Quick left
\renewcommand{\r}{\right}       % Quick right
\newcommand{\f}{\frac}          % Quick fraction
\newcommand{\Z}{\mathcal{Z}}    % GC partition function
\newcommand{\D}{\mathcal{D}}    % Swirly D for path integrals
\newcommand{\fphi}{\tilde{\phi}}% Fourier space phi
\newcommand{\fh}{\tilde{h}}     % Fourier space h
\newcommand{\fxi}{\tilde{\xi}}  % Fourier space xi
\newcommand{\A}{\rho_A}         % Density of species A
\newcommand{\B}{\rho_B}         % Density of species B
\newcommand{\ham}{\mathcal{H}}  % Hamiltonian (classical)
\newcommand{\q}{\mathbf{q}}     % Phase space coordinates
\newcommand{\p}{\mathbf{p}}     % Phase space momenta

% Plane wave + and -
\newcommand{\pwp}{e^{\frac{i\mathbf{p}\cdot\mathbf{q}}{\hbar}}}
\newcommand{\pwm}{e^{-\frac{i\mathbf{p}\cdot\mathbf{q}}{\hbar}}}

% Classical Trace
\newcommand{\trace}[1]{\mathrm{Tr}\left( #1 \right)}

% Average
\newcommand{\mean}[1]{\left\langle #1 \right\rangle}

\listfiles

%%%%%%%%%%%%%%%%%%
% Document Begin %
%%%%%%%%%%%%%%%%%%

\begin{document}

%  Front Matter 
\maketitle
\begin{romanPagenumber}{2}

\SetDedicationName{\MakeUppercase{Dedication}}
\SetDedicationText{My dedication....}
\Dedication

\SetAcknowledgeName{\MakeUppercase{Acknowledgements}}
\SetAcknowledgeText{Acknowledgments, if included, must be written in complete sentences.  Do not use direct address.  For example, instead of Thanks, Mom and Dad!, you should say I thank my parents. }
\Acknowledge


%  English Abstract
\SetAbstractEnName{\MakeUppercase{Abstract}}
\SetAbstractEnText{ Abstract in English and French are required. The text of the abstract in English begins here.}
\AbstractEn

%  French Abstract  
\SetAbstractFrName{\MakeUppercase{ABR\'{E}G\'{E}}}
\SetAbstractFrText{ The text of the abstract in French begins here.  }
\AbstractFr

\TOCHeading{\MakeUppercase{Table of Contents}}
\LOTHeading{\MakeUppercase{List of Tables}}
\LOFHeading{\MakeUppercase{List of Figures}}
\tableofcontents 
\listoftables 
\listoffigures 

\end{romanPagenumber}

% -- Main Matter --
\chapter{Introduction}

\chapter{Fundamentals}
In this chapter we'll describe the fundamental physics behind the phase field crystal theory.
Like many physical theories, it is derived using a set of successive approximations.
Each approximation yield a new theory that is more narrow in scope, but more tractible to either analytic or numerical analysis.

PFC is ultimately a thermodynamic theory and as such it makes connection to fundamental, microscopic physics by way of statistical mechanics.
Statistical mechanics tied the macroscopic observables to microscopic phenomena with a probabilistic approach.
The basic assumption at this level is that, if a system is sufficiently complex, there are circumstances under which its statistical behaviour becomes relatively simple.
Now for systems with macroscopic ($\mathcal{O}(10^{23})$) amount of particles this is almost always the case and so for this large systems we can often compute the macroscopic or thermodynamic observables using a statistical approach instead of solving the microscopic equations of motion.

At the fundamental level, our systems of interest are governed by quantum mechanics and so we might use the theory of quantum statistical mechanics to attempt to compute the thermodynamic observables of our system.
We will see that for our systems of interest that the quantum statistical theory is quite intractible, but with an approximation we can treat our system in the \textit{semi-classical limit}.

In semi-classical limit, we can talk in a robust way about the structure of the denisty field, using the Classical Density Functional Theory (CDFT) framework.
While CDFT supplies the correct setting to discuss the density, it is again rarely feasible to perform exact calculations.

Finally, we'll see that an approximation of the exact CFDT free energy functional will yield the PFC theory that is amenable to both analytic and numerical analysis.

\section{Statistical Mechanics in the Semi-classical limit}

Although the quantum statistical mechanics picture gives us a link between the microscopic and macroscopic reality of thermodynamics systems, it still contains too much detail for many systems of interest.
For instance, for many systems of interest, the precise bosonic or fermionic nature of the particles in the system has little consequence on the thermodynamic properties.
We can ignore some of these quantum mechanical details by looking at statistical mechanics in the \textit{semi-classical limit}.

For the sake of clarity, we'll look at a system of $N$ identical particles in the canonical ensemble but generalization to multi-component systems and other ensembles is straight forward. 
We start with the definition of the partition function for a system of many particles,  
\begin{equation}
    Z = Tr(e^{\beta \hat{H}}),
\end{equation} 
where, $\hat{H} = \frac{\vert\mathbf{p}\vert^2}{2m} + V(\mathbf{q})$ and $\mathbf{p} = (p_1, p_2, ...p_N)$ is the vector of particle momenta. 
$\mathbf{q}$ is similarly defined for the particle positions.
Wigner \cite{PhysRev.40.749},and, shortly after, Kirkwood \cite{PhysRev.44.31} showed that the partition function could be expanded in powers of $\hbar$, facilitating the calculation of both a classical limit and quantum corrections to the partition function.
Their method, the Wigner-Kirkwood expansion, involves evaluating the trace operation over a basis of plain wave solutions,
\begin{equation}
	\mathcal{Z}(\beta) = \int 
		\frac{\mathrm{d}\mathbf{q} \mathrm{d}\mathbf{p}}{(2\pi \hbar)^N}
		e^{-\frac{i\mathbf{p}\cdot\mathbf{q}}{\hbar}}
		e^{-\beta \hat{H}}
		e^{\frac{i\mathbf{p}\cdot\mathbf{q}}{\hbar}} = \int d\Gamma I(p, q),
\end{equation}
Where, $d\Gamma$ is the phase space measure $d\mathbf{p}d\mathbf{q}/(2\pi\hbar)^N$.
To compute the integrand, $I(p, q)$, we follow Uhlenbeck and Bethe \cite{Uhlenbeck1936729} and first compute its derivative,
\begin{equation}
	\frac{\partial I(p, q)}{\partial \beta} = -\pwp\hat{H}\pwm I(p, q).
\end{equation}
If we then make a change of variables, $I(q, p) = e^{-\beta H}W(q, p)$ use the explicit form of the Hamiltonian we arrive at a partial differential equation for $W$.
\begin{equation}
	\frac{\partial W}{\partial \beta} = \frac{\hbar^2}{2} \left(
		\nabla_{\mathbf{q}}^2 - 
		\beta(\nabla_{\mathbf{q}}^2V) + 
		\beta^2(\nabla V)^2 -
		2\beta(\nabla_{\mathbf{q}} V)\cdot\nabla_{\mathbf{q}} + 
		2 \frac{i}{\hbar}\mathbf{p}\cdot(\nabla_{\mathbf{q}} - \beta\nabla_{\mathbf{q}})
	\right)W(q, p)
\end{equation}
The solution can be written as a power series in $\hbar$, $W = 1 + \hbar W_1 + \hbar W_2 + ...$.
This creates a power series expansion for the partition function as well,
\begin{equation}
    \mathcal{Z} = (1 + \hbar \langle W_1 \rangle + \hbar^2 \langle W_2\rangle + ...) 
    \int d\Gamma e^{\beta\mathcal{H}}.
\end{equation}
Where the average, $\langle \cdot \rangle$, denotes the the classical average, 
\begin{equation}
    \langle A(p, q) \rangle = \frac{1}{\mathcal{Z}} 
        \int d\Gamma A(p, q) e^{-\beta \mathcal{H}}.
\end{equation}
For the sake of brevity we'll simply quote solution to second order, but details can be found in Landau and Lifshitz \cite{LANDAU198079}.
Interestingly, the first order term is exactly zero.
\begin{gather}
    \langle W_1 \rangle = 0 \\
    \langle W_2 \rangle = - \f{\beta^3}{24 m} \l\langle \left\vert\nabla_{\mathbf{q}}V\right\vert^2 \r\rangle
\end{gather}
In terms of the free energy, for example, the corrections to second order would be, 
\begin{equation}{\label{quantumcorr}}
    \mathcal{F} = \mathcal{F}_{classical} + \f{\hbar^2\beta^2}{24m}
        \l\langle \l\vert \nabla_{\mathbf{q}} V(\mathbf{q}) \r\vert^2 \r\rangle.
\end{equation}

There are a few things to note about this finding, first of all the correction inversely proportional to both the temperature and the particle mass.
For copper at room termperature, for instance, the prefactor $\hbar^2\beta^2/(24 m)$ is $\mathcal{O}(10^{-4})$. 
The correction is also proportional to the mean of the squared force felt by each particle. So high density materials will have a higher quantum correction because they sample the short-range repulsive region of the pair potential more than their low density counter parts.

\subsection{Indistinguishability}

There is an important distinction to be made between the quantum theory and the theory in the semi-classical limit.
The integral over phase space of the partition function must only take into account the \textit{physically different} states of the system.  
In the quantum theory this is acheived by tracing over any orthonormal basis of the Hilbert space, but in the classical theory we need to be careful not to double count states when identical particles are in the theory.
Exchange of two identical particles does not result in a physically different state and thus this state should only be considered only once in the sum over states in the partition function.
More precisely, we should write the classical partition function as,

\begin{equation}
    \mathcal{Z} = \int^\prime d\Gamma e^{-\beta \mathcal{H}},
\end{equation}
Where the primed integral denotes integration only over the physically distinct states. In the common case of $N$ identical particles, the phase space integral becomes, 
\begin{equation}
    \int^\prime d\Gamma \rightarrow \f{1}{N!}\int d\Gamma
\end{equation}

Aggregating our results, we can write the partition function in the semi-classical limit as,
\begin{equation}
    \mathcal{Z}(\beta) = \frac{1}{N!}\int d\Gamma e^{-\beta \mathcal{H}},
\end{equation}
Or, in the grand canonical ensemble,
\begin{equation}
    \Xi(\mu, \beta) = \sum_{N = 0}^\infty \f{e^{\beta\mu N}}{N!} \int d\Gamma e^{-\beta \mathcal{H}}
\end{equation}

Of course, this is exactly the form taught in introductory courses on statistical mechanics and derived by Gibbs\footnote{The $\hbar$ in Gibbs' formula was justified on dimensional grounds and was simply a scaling factor with units of action ($J\cdot s$)}prior to any knowledge of quantum mechanics \cite{Gibbs}.
The key insight here is to understand in a contolled way when this approximation is accurate and the magnitude of the next quantum correction is as seen in equation \ref{quantumcorr}.

\section{Classical Density Functional Theory}




\chapter{Classical Density Functional Theory of Freezing}
\label{dft_of_freezing}

The classical density functional theories derived in chapter
\ref{fundamentals} was first established to study inhomogenous fluids.
Interestingly, one can think of the solid state as an especially extreme
case of an inhomogeneous fluid \cite{HANSEN-CH6}.  In this case, we can
use CDFT to study the circumstances under which the density field develops
long range periodic solutions (ie., solidification).  While not expressed
in precisely this language, this idea dates back as far as 1941 with the
early work of Kirkwood and Monroe \cite{KIRKWOOD_MONROE41} and was later
significantly refined by Youssof and Ramakrishnan \cite{RAMAKRISHNAN79}.


%%%%%%%%%%%%%%%%%%%%%%%%%%%%%%%%
\section{Amplitude Expansions} %
%%%%%%%%%%%%%%%%%%%%%%%%%%%%%%%%

To explore the problem of solidification, we begin with the approximate grand
potential established in equation \ref{cdft_grand_potential} with the external
potential, $\phi(r)$, set to zero,
%
\begin{equation}
    \beta \Delta \Omega[\rho(r)] =
        \int dr \l\lbrace 
            \rho(r)\ln\l(\f{\rho(r)}{\rho_0}\r) - \Delta\rho(r)\r\rbrace
        - \f{1}{2} \Delta\rho(r) \ast C^{(2)}_0(r, r^\prime)
            \ast \Delta\rho(r^\prime).
\end{equation}
%
Scaling out a factor of $\rho_0$ we can rewrite the grand potential in terms of
a dimensionless reduced density, $n(r) \equiv (\rho(x) - \rho_0)/\rho_0$,
%
\begin{equation}
    \label{gp}
    \f{\beta \Delta \Omega[n(r)]}{\rho_0} =
        \int dr \l\lbrace 
            (1 + n(r))\ln\l(1 + n(r)\r) - n(r)\r\rbrace
        - \f{1}{2} n(r) \ast \rho_0 C^{(2)}_0(r, r^\prime) \ast n(r^\prime).
\end{equation}
%
To describe the density profile in the solid state we can expand the density
in a plane waves,
%
\begin{equation}
    \label{expansion}
    n(r) = \bar{n} + \sum_{\mathbf{G}} \xi_{\mathbf{G}} e^{i \mathbf{G} r}.
\end{equation}
%
Where, $\l\lbrace \mathbf{G} \r\rbrace$, is the set of reciprocal lattice
vectors in the crystal lattice and the amplitudes, $\xi_\mathbf{G}$, serve as
order parameters for freezing. In the liquid phase all amplitudes are zero and
the average density is uniform, while in the solid phase there are finite
amplitudes that describe the periodic profile of the crystal lattice. Setting
the liquid at the melting point as our reference fluid we find that, at the 
melting point, $\bar{n}$ is zero for the liquid and is the fractional density
change of solidification, $(\rho_s - \rho_l)/\rho_l$, in the solid phase.

The amplitudes are constrained by the point group symmetry of the lattice.
Grouping the amplitudes of symmetry-equivalent reciprocal lattice vectors
together we can write the density profile as,
%
\begin{equation}
    \label{amplitudes}
    n(r) = \bar{n}
         + \sum_\alpha \l\lbrace
            \xi_\alpha \sum_{\lbrace\mathbf{G}\rbrace_\alpha}
                e^{i\mathbf{G} \cdot \mathbf{x}}\r\rbrace,
\end{equation}
%
Where $\alpha$ is a label running over sets of symmetry-equivalent reciprocal
lattice vectors.

If we insert equation \ref{amplitudes} into equation \ref{gp} and integrate
over the unit cell we find,
%
\begin{align}
    \label{amplitude_gp} 
    \f{\beta \Delta\Omega_{cell}}{\rho_0} &=  \int_{cell} 
        dr \l\lbrace (n(r) + 1)\ln\l(n(r) + 1\r) - n(r) \r\rbrace \nonumber \\
    &- \f{1}{2} \l[\bar{n} ^ 2 \rho_0\tilde{C}^{(2)}_0(0) + \sum_\alpha\rho_0\tilde{C}^{(2)}_0
            (\mathbf{G}_\alpha) \lambda_\alpha\vert\xi_\alpha\vert^2\r],
\end{align}
%
Where $\lambda_\alpha$ is the number of reciprocal lattice vectors in the set
$\alpha$ and $\tilde{C}^{(2)}_0(k)$ is the Fourier transform of the direct
correlation function of the reference fluid. The first term in equation
\ref{amplitude_gp} is convex in all of the amplitudes with a minimum at zero.
It follows that solidification must occur when the direct correlation function
at the reciprocal lattice vectors, $\tilde{C}^{(2)}_0(\mathbf{G}_\alpha)$, is
large enough to stabilize a finite amplitude by creating a new minimum away
from zero.

Furthermore, equation \ref{amplitude_gp} suggests that this transition depends
only on a set of parameters, $\rho_0 \tilde{C}^{(2)}_0(\mathbf{G}_\alpha)$,
that are material independent.  That is to say, once we specify the symmetry of
the lattice a liquid will solidify into (eg.  face-centred-cubic), all
materials that undergo this transition should share these parameters at the
melting point. This seems to be the case for a variety of materials. For
instance, for many liquids solidifying into face-centred-cubic (fcc) lattices,
these parameters for the [111] and [311] reciprocal lattice vectors are 0.65
and 0.23 respectively. Similarly, for a spectrum of liquids solidifying into
body-centred-cubic (bcc) lattices these parameters  for the [110] and [211]
reciprocal lattice vectors are approximately 0.66 and 0.12 respectively.

As seen in table \ref{table:ramakrishnan_argon} and table
\ref{table:ramakrishnan_sodium} theoretical results from this approach match
very closely to experimental values. In spite of the successes of density
functional approach pioneered by Youssof and Ramakrishnan there are limits
to this framework.

\begin{table}[]
    \center
    \begin{tabular}{l c c c}
        \hline 
        Theory & $\tilde{C}(\mathbf{G}_{[111]})$ & $\tilde{C}(\mathbf{G}_{[311]})$ & $\bar{n}$ \\ 
        \hline
        I & 0.95 & 0.0 & 0.074 \\
        II & 0.65 & 0.23 & 0.270 \\
        III & 0.65 & 0.23 & 0.166 \\
        Experiment & 0.65 & 0.23 & 0.148\\
        \hline
    \end{tabular}
    \caption[Freezing parameters for Argon]{Freezing parameters for Argon (fcc)
    taken from \cite{RAMAKRISHNAN79}.  Theory I uses one order paramter, theory
    II uses two order parameters and theory III uses two order parameters and
    expands to third order in the free energy. $\eta$ is the fractional density
    change of solidification 
    $(\rho_s - \rho_l) / \rho_l)$.}
    \label{table:ramakrishnan_argon}
\end{table}

\begin{table}[]
    \center
    \begin{tabular}{l c c c}
        \hline 
        Theory & $\tilde{C}(\mathbf{G}_{[110]})$ & $\tilde{C}(\mathbf{G}_{[211]})$ & $\bar{n}$ \\ 
        \hline
        I           & 0.69 & 0.00 & 0.048 \\
        II          & 0.63 & 0.07 & 0.052 \\
        III         & 0.67 & 0.13 & 0.029 \\
        Experiment  & 0.65 & 0.23 & 0.148\\
        \hline
    \end{tabular}
    \caption[Freezing parameters for Sodium]{Freezing parameters for Sodium
    (bcc) taken from \cite{RAMAKRISHNAN79}.  Theory I uses one order paramter,
    theory II uses two order parameters and theory III uses two order
    parameters and expands to third order in the free energy. $\eta$ is the
    fractional density change of solidification 
    $(\rho_s - \rho_l) / \rho_l)$.}
    \label{table:ramakrishnan_sodium}
\end{table}

Many of the mechanical properties of solids are due to the way in which they
deviate from the perfect crystalline lattice: the microstructure. Grain
boundaries, vacancies, dislocations and second phase particles all play
critical roles in determining the mechanical properties of solids. A simple
example of this is the Hall-Petch effect which states that the yield stress of
a material increases with decreasing grain size,
%
\begin{equation}
    \sigma = \sigma_0 + k_y d^{-1/2},
\end{equation}
%
where, $d$, is the average grain diameter, $\sigma$ is the yield stress,
$\sigma_0$ is yield stress of a single crystal sample and $k_y$ is the
strengthening coefficient.

Microstructure plays key roles not only in the mechanical properties but also
the kinetic pathways of certain phase transformations. For instance,
dislocations can act to catalyze precipication in binary alloys [cite Vahid].

The problem that we face is then, how do we examine these defects and
microstructural elements but retain some of the successful aspects of density
functional approach? One way to think about to consider the microstructure in
the solid state is that it is an artifact of not having fulling reached
equilibrium. Real materials are solidified over a finite time and therefore
haven't fully reached equilibrium. As a result, we should study the pathway to 
equilibrium to gain insight into the origin of microstructure. 

%%%%%%%%%%%%%%%%%%%%%%%%%%%%%%%%%%%%%%%%%%%%%
\section{Dynamic Density Functional Theory} %
%%%%%%%%%%%%%%%%%%%%%%%%%%%%%%%%%%%%%%%%%%%%%

Using techniques from non-equilibrium statistical mechanics we can extend
the density functional approach to a dynamic model. To start we illustrate the 
non-equilibrium method schematically. Consider a non-equilibrium probability 
distribution over phase space, $f(\q, \p; t)$. As a function over phase space,
its equation of motion is a simple result of classical mechanics,
%
\begin{equation}
    \label{cm} 
    \f{d f}{dt} = \l\lbrace f, \mathcal{H} \r\rbrace + \f{\partial f}{\partial t}.
\end{equation}
%
Where, $\l\lbrace \cdot, \cdot \r\rbrace$, denotes the Poisson bracket,
%
\begin{equation}
    \l\lbrace f, g \r\rbrace = \sum_{i = 0}^N \f{\partial f}{\partial q_i}
        \f{\partial g}{\partial p_i} - \f{\partial g}{\partial q_i}
        \f{\partial f}{\partial p_i}.
\end{equation}
%
Of course, the distribution must remain normalized in time and therefore the 
total time derivative must be zero,
%
\begin{equation}
    \int d\q d\p\, f(\q, \p; t) = 1 \rightarrow \f{d f}{dt} = 0.
\end{equation}
%
Accounting for this conservation law in equation \ref{cm}, the resulting
equation of motion is called the \textit{Liouville Equation},
%
\begin{equation}
    \label{liouville} 
    \f{\partial f}{\partial t} = - \l\lbrace f , \mathcal{H} \r\rbrace
\end{equation}
%
Under appropriate conditions the probability distribution, under the action of
the Liouville Equation, will decay to a stable fixed point $f_{eq}(\q, \p)$ we
call equilibrium,
%
\begin{equation}
    \lim_{t \rightarrow \infty} f(\q, \p; t) = f_{eq}(\q, \p)
\end{equation}
%

Using the non-equilibrium probability distribution, we can also discuss
non-equilibrium averages of the density profile and their associated equations
of motions. The non-equilibrium density is written in analogy with equation
\ref{mean_density} by taking of the classical trace of the density operator
over with the non-equilibrium distribution,
%
\begin{equation}
    \rho(x, t) = \mean{\hat{\rho}(x; \q)}_{ne} =
        \trace{\hat{\rho}(x; \q) f(\q, \p, t)}.
\end{equation}
%
Where, $\mean{\cdot}_{ne}$, denotes the non-equilibrium average. Just as the
non-equilibrium probability distribution is driven to equilibrium by the
Liouville Equation, so too is the density profile by its own equation of
motion.

A variety of equations of motion for the density field are known.  For
instance, we can consider the Navier-Stokes equations of hydrodynamics to one
such equation of motion. If we restrict ourselves to diffusion limited
circumstances, we may derive a much simplier equation of motion. To acheive
this result we use the projection operator method, and assume that the 
density operator is the only relevant variable. Quoting the result from
\cite{ESPANOL09} we find,
%
\begin{equation}
    \label{eq:mean_eom}
    \f{\partial \rho(x, t)}{\partial t} = 
        \nabla \cdot \integrate{r^\prime} \mathbf{D}(r, r^\prime, t) 
        \cdot \nabla^\prime \f{\d \F[\rho]}{\d \rho(x^\prime, t)},
\end{equation}
%
Where, $\mathbf{D}(r, r^\prime, t)$, is the diffusion tensor,
%
\begin{equation}
    \mathbf{D}(r, r^\prime, t) = \int_0^\infty \mathrm{d}\tau^\prime\,
        \trace{f(\q, \p, t)\hat{\mathbf{J}}(r, 0)
        \hat{\mathbf{J}}(r^\prime, \tau^\prime)},
\end{equation}
%
in which $\mathbf{J}(r, t)$ is the density flux is,
%
\begin{equation}
    \hat{\mathbf{J}}(r, t) \equiv 
        \sum_i^N \frac{p_i}{m_i} \d(q_i - r).
\end{equation}

Should we use assume 

{
    \color{ForestGreen}

If we restrict ourselves to the case where
transport is diffusion limited and we assume that the dynamic pair correlation
function is the same as the equilibrium pair correlation we can express the
density equation of motion as,
%
\begin{equation}
    \label{mean_eom}
    \f{\partial \rho(x, t)}{\partial t} = 
        \nabla \cdot \l[
            D_0 \rho(x, t) \nabla \l(\f{\d \F[\rho]}{\d \rho(x, t)}\r)
        \r].
\end{equation}
%
Where, $D_0$ is the diffusion constant. We can also express the dynamics of
the density field in the form of a Langevin equation for the density operator,
%
\begin{equation}
    \label{langevin_eom}
    \f{\partial \hat{\rho}(x, t)}{\partial t} =
        \nabla \cdot \l[
            D_0 \hat{\rho}(x, t) \nabla \l(\f{\d \F[\hat{\rho}]}{\d \hat{\rho}}\r)
        \r] + \xi(x, t).
\end{equation}
%
Where, $\mathbf{\xi}(x, t)$ is a Gaussian random driving force with zero mean
and variance of,
%
\begin{equation}
    \mean{\xi(x, t)\xi(x^\prime, t^\prime)} = -2 \nabla \cdot \l[ D_0 \rho(x,t) 
        \nabla \d(x - x^\prime) \d(t - t^\prime)
    \r],
\end{equation}
%
Due to a generalized Einstein relation\footnote{See Appendix \ref{noise} for
details on generalized Einstein relations for nonlinear Langevin equations} for
the Langevin equation in equation \ref{langevin_eom}.

Equation \ref{mean_eom} and \ref{langevin_eom} were first derived by Marconi
and Tarazona by considering an ensemble of interacting Brownian particles, but
other derivations are possible including the projection operator method [cite
Espanol].  Equations \ref{mean_eom} and \ref{langevin_eom} fall under the
heading of \textit{Dynamic Density Functional Theory} (DDFT) or sometimes
\textit{Time dependent density functional theory} (TD-DFT) though we'll use the
former throughout this text.
}

Unfortunately, if we were to use the approximate free energy functional
established in equation \ref{cdft_free_energy} in the dynamic density
functional theory of equation \ref{mean_eom} or \ref{langevin_eom} we would
face a major impedement: the solid state solutions of the density functional
theory approach yield sharply peaked solutions at the position of the atoms in
the lattice. While this is realistic, they are a major challenge for numerical
algorithms. The challenges are two fold: first, these sharp peaks require a fine mesh 
to be resolved resulting in large memory requirement to simulate domains 
of any non-trivial scale and second, lineary stability analysis of most algorithms 
slave the scale of the time step to the scale of the grid spacing so only small
time steps can be taken on a fine mesh.

%%%%%%%%%%%%%%%%%%%%%%%%%%%%%%%%%%%%%%
\section{Phase Field Crystal Theory} %
%%%%%%%%%%%%%%%%%%%%%%%%%%%%%%%%%%%%%%

The phase field crystal theory (PFC) presents a solution to the numerical difficulties
faced by DDFT methods by approximating the free energy in such a say as to retain the
basic features of the theory with a smoother solid state solution. Starting the 
approximate free energy functional of equation \ref{cdft_free_energy} we proceed as
previously by scaling out a factor of the reference density and changing variables
to a dimensionless density $n(r) = (\rho(r) - \rho_0) / \rho_0$,
%
\begin{equation}
    \f{\beta \F[n(r)]}{\rho_0} = 
        \int dr \l\lbrace (n(r) + 1) \ln( n(r) + 1) - (1 - \beta\mu)n(r) \r\rbrace
        - \f{1}{2} n(r) \ast \rho_0 C^{(2)}_0(r, r^\prime) \ast n(r^\prime).
\end{equation}
%
We then Taylor expand the logarithm about the reference density or equivalently
$n(r) = 0$, to fourth order,
%
\begin{equation}
    \label{pfc_free_energy} 
    \f{\beta \F[n(r)]}{\rho_0} =
        \int dr \l\lbrace \f{n(r)^2}{2} - \f{n(r)^3}{6} + \f{n(r)^4}{12} \r\rbrace
        -\f{1}{2} n(r) \ast \rho_0 C^{(2)}_0(r, r^\prime) \ast n(r^\prime).
\end{equation}
%
Where the linear term has been dropped because it leaves the equations of
motion invariant.  Most phase field crystal theories also use a simplified
equation of motion as well,
%
\begin{equation}
    \label{pfc_eom}
    \f{\partial n(r, t)}{\partial t} = M \nabla^2 \l(\f{\d \F[n(r)]}{\d n(r)}\r).
\end{equation}
%


\chapter{Fake}
In a many body system in which particles interactions are independent of their velocities, we may split contributions to the free energy into two parts: the ideal and the excess.

\subsubsection{The ideal component of the free energy}
The ideal component comes from the kinetic energy term in the Hamiltonian and is known exactly.

\begin{equation}
    \beta\F_{id}[\rho] = \int dr \rho(r) \l\lbrace\ln\right(\Lambda^3\rho(r)\left) - 1\r\rbrace
\end{equation}
Where,
\begin{description}[labelindent=10pt, labelsep=10pt]
\item[$\Lambda^3$] is the thermal DeBroglie volume
\item[$\rho(r)$] is the number density
\item[$\beta$] is the inverse temperature ($1/k_bT$)
\end{description}

\subsubsection{Expansion of the excess free energy}
The excess component comes from the interaction term in the Hamiltonian and is often not known exactly and must be modelled.
A common technique for modelling the excess free energy is to expand it around uniform fluid reference state.
This reference state is characterized by a number density, $\rho_0$, and chemical potential $\mu_0$.

\begin{equation}
    \beta\F_{ex}[\rho] = \beta\F_{ex}^0 +
    \beta\int dr \left.\frac{\d\F_{ex}}{\d\rho}\right\vert_{\rho = \rho_0} \Delta\rho
    + \beta\f{1}{2}\int dr \int dr^\prime \Delta\rho(r^\prime)\l.\f{\d^2\F_{ex}}{\d\rho(r)\d\rho(r^\prime)}\r\vert_{\rho=\rho_0}\Delta\rho(r^\prime) + ...
\end{equation}

Where,
\begin{description}[labelindent=10pt, labelsep=10pt]
    \item[$\Delta\rho$] is difference from reference density ($\rho(r) - \rho_0$)
    \item[$\F_{ex}^0$] is the free energy of the reference state
\end{description}

The excess free energy is the generating function of direct correlation functions $C^(n)(r_1, ..., r_n)$.
In particular this means that direct correlation functions can be written as,

\begin{equation}
    C^{(n)}(r_1, ..., r_n) = -\beta \f{\d^{(n)}\F_{ex}[\rho]}{\d\rho(r_1)...\d\rho(r_n)},
\end{equation}
and our previous expansion can be rewritten in terms of these direct correlation functions.

\begin{equation}
    \beta\F_{ex}[\rho] = \beta\F_{ex}^0 - \int dr C^{(1)}_0(r) \Delta\rho
    + \beta\f{1}{2}\int dr \int dr^\prime \Delta\rho(r^\prime)C^{(2)}_0(r, r^\prime)\Delta\rho(r^\prime) + ...
\end{equation}

In the absence of an external field the single particle direct correlation function of the reference system is simply $\beta\mu^{ex}_0$.

\subsubsection{Total Free Energy}
If we now add in the ideal contribution to the free energy and take advantage of the fact that the excess chemical potential of the reference fluid can be written as the total chemical potential minus the ideal contribution,

\begin{equation}
    \mu^{ex}_0 = \mu_0 - \mu^{id}_0 = \mu_0 - k_bT\ln(\Lambda^3\rho_0).
\end{equation}
to find an expression for the total free energy:

\begin{equation}
    \beta\F[\rho] = \beta\F_0 + \int\,dr \l\lbrace \Delta\rho \ln\l(\f{\Delta\rho}{\rho_0}\r) - (1 - \mu_0)\Delta\rho \r\rbrace
    - \f{1}{2}\int\,dr\int\,dr^\prime \Delta\rho(r) C^{(2)}_0(r, r^\prime) \Delta\rho(r^\prime)
\end{equation}

\subsubsection{Smooth atom approximation and the PFC Free Energy}

To construct the phase-field crystal free energy we assume that the density fluctations, $\Delta\rho(r)$, are small and expand the logarithm term to quartic order in the fluctuations.
Furthermore, we nondimensionalize the free energy by scaling out the reference density, $\rho_0$, and changing variables to $n(r) = \Delta\rho(r)/\rho_0$.

\begin{equation}
    \f{\beta\F[n]}{\rho_0} = \f{\beta\F_0}{\rho_0} +
    \int dr \mu_0 \rho_0 n(r) + \f{n^2(r)}{2} - \f{n^3(r)}{6} + \f{n^4(r)}{12}
    -\f{\rho_0}{2} \int dr \int dr^\prime n(r) C^{(2)}_0(r, r^\prime) n(r^\prime)
\end{equation}

At this point we note that the constant and linear terms in the free energy can be removed without changing the properties of the functional and we are left with the following minimal free energy functional:

\begin{equation}
    \f{\beta\F}{\rho_0} = \int dr \f{n^2(r)}{2} - \f{n^3(r)}{6} + \f{n^4(r)}{12}
    -\f{1}{2} \int dr \int dr^\prime n(r) C^{(2)}(r, r^\prime) n(r^\prime).
\end{equation}

Note, that by convention a factor of the reference density is absorbed into the pair correlation function in the PFC free energy functional.


\chapter{Dynamics}
\input{chapters/ddft}

\chapter{Applications}
asdfasdf

% -- Back Matter --
\SetAppendixName{Appendix}
\SetAppendixText{Text!}
\ETDAppendix{Noise in Nonlinear Langevin Equations}{
    When using Langevin equations to study non-equilibrium statistical mechanics the noise strength can be linked to the transport coefficients through a generalization of the Einstein relation, $D = \mu k_bT$. The typical strategy for deriving such a relationship is to evaluate the equilibrium pair correlation function by two separate methods: the equilibrium partition functional and the equation of motion.

While the equilibrium partition functional gives pair correlation through the typical statistical mechanical calculation, the equation of motion can be used to derive a dynamic pair correlation function that must be equal to the equilibrium pair correlation function in the long time limit.

In what follows we'll look at how to formulate a generalized Einstein relation from a generic Langevin equation and then calculate two specific examples using Model A dynamics and a $phi^4$ theory and Time Dependent Density Functional Theory (TDDFT) and a general Helmholtz free energy.

\section{Generalized Einstein Relations in an Arbitrary Model}

We start by considering a set of microscopic observables, $a_i(r, t)$, that are governed by a nonlinear Langevin equation,

\begin{equation}
	\f{\partial \mathbf{a}(r, t)}{\partial t} = F[\mathbf{a}(r,t)] + \boldsymbol{\xi}(r,t).
\end{equation}

Where, $\mathbf{a}$, denotes a vector of our fields of interest. These microscopic equation of motion may have been derived from linear response, projection operators or some other non-equilibrium formalism. We assume that the random driving force, $\boldsymbol{\xi}(r, t)$ is unbaised, Gaussian noise that is uncorrelated in time.

\begin{gather}
	\l\langle \boldsymbol{\xi}(r,t)\r\rangle = 0 \\
	\l\langle \boldsymbol{\xi}(r,t)\boldsymbol{\xi}^\dagger(r^\prime,t^\prime)\r\rangle =
	\mathbf{L}(r, r^\prime)\d(t-t^\prime)
\end{gather}

We wish to constrain the form of the covariance matrix, $\mathbf{L}$ of $\boldsymbol{\xi}$ by demanding that the solution to the Langevin equation eventually decays to equilibrium and that correlations in equilibrium are given by Boltzmann statistics.

We begin by linearizing the equation of motion about an equilibrium solution, $\mathbf{a}(r, t) = \mathbf{a}_{eq}(r) + \hat{\mathbf{a}}(r, t)$.

\begin{equation}
	\f{\partial \hat{\mathbf{a}} (r, t)}{\partial t} = \mathbf{M}(r, r^\prime) \ast \hat{\mathbf{a}}(r^\prime, t) + \boldsymbol{\xi}(r, t)
\end{equation}

Where, $\ast$ denotes an inner product and integration over the repeated variable. eg:

\begin{equation}
	\mathbf{M}(r, r^\prime)\ast \hat{\mathbf{a}}(r^\prime) = \sum_j \int\,dr^\prime M_{ij}(r, r^\prime) \hat{a}_j(r^\prime).
\end{equation}

We can formally solve our linearized equation of motion.

\begin{equation}
	\hat{\mathbf{a}}(r, t) = e^{\mathbf{M}(r, r^\prime)t}\ast\hat{\mathbf{a}}(r^\prime, 0) + \int_0^t d\tau\, e^{\mathbf{M}(r, r^\prime)(t-\tau)} \ast \boldsymbol{\xi} (r^\prime, \tau)
\end{equation}

We can use this formal solution to evaluate the equilibrium pair correlation function.

\begin{align}
	\l\langle \hat{\mathbf{a}}(r, t)\hat{\mathbf{a}}^\dagger(r^\prime, t^\prime) \r\rangle &= e^{\mathbf{M}(r, r_1)t}\ast\l\langle \hat{\mathbf{a}}(r_1, 0)\hat{\mathbf{a}}^\dagger(r_2, 0) \r\rangle \ast e^{\mathbf{M}^\dagger(r^\prime, r_2)t^\prime} \nonumber \\
	&+ \int_0^t \int_0^{t^\prime}d\tau d\tau^\prime\, e^{\mathbf{M}(r, r_1)(t-\tau)}\ast\l\langle \boldsymbol{\xi}(r_1, 0)\boldsymbol{\xi}^\dagger(r_2, 0) \r\rangle \ast e^{\mathbf{M}^\dagger(r^\prime, r_2)(t^\prime-\tau^\prime)}
\end{align}

It is important to note that every eigenvalue of $\mathbf{M}$ must be negative for our solution to decay to equilibrium in the long time limit (eg. $lim_{t\rightarrow\infty}\hat{\mathbf{a}}(r, t) = 0$) and as such the first term in our dynamic correlation function won't contribute to the equilibrium pair correlation. This is as we might expect as the first term holds the contributions to the correlation function from the initial conditions. The second term can be evalutated by substituting the noise correlation and evaluating the delta function.

\begin{equation}
	\mathbf{\Gamma}(r, r^\prime) = \lim_{t\rightarrow\infty} \l\langle \hat{\mathbf{a}}(r, t)\hat{\mathbf{a}}^\dagger(r^\prime, t) \r\rangle =
	\int_0^\infty dz \,e^{\mathbf{M}(r, r_1)z}\ast\mathbf{L}(r_1, r_2)\ast e^{\mathbf{M}^\dagger(r^\prime, r_2)z}
\end{equation}

Considering the product $\mathbf{M}(r, r_1)\ast\mathbf{\Gamma}(r_1, r^\prime)$ and performing an integration by parts gives the final generalized Einstein relation.

\begin{equation}
	\mathbf{M}(r, r_1)\ast\mathbf{\Gamma}(r_1, r^\prime) + \mathbf{\Gamma}(r, r_1)\ast\mathbf{M}^\dagger(r_1, r^\prime) = -\mathbf{L}(r, r^\prime)
\end{equation}

\section{Fluctuation Dissipation in Model A}

As a first example of calculating an Einstein relation consider the following free energy functional under non-conservative, dissipative dynamics.

\begin{gather}
\beta \F[\phi] = \int dr \left\lbrace \f{1}{2}\vert \nabla \phi(x) \vert^2 + \f{r}{2}\phi^2(x) + \f{u}{4!}\phi^4(x)  + h(x)\phi(x)\right\rbrace \\
\f{\partial \phi(x,t)}{\partial t} = -\Gamma \left(\f{\delta \beta \F[\phi]}{\delta \phi(x)}\right) + \xi(x, t)
\end{gather}

The random driving force, $\xi$, is Gaussian white noise with some scalar noise strength $\lambda$.

\begin{align}
\l\langle \xi (x, t) \r\rangle &= 0 \\
\l\langle \xi (x, t) \xi(x^\prime, t^\prime) \r\rangle  &= \lambda \delta(x - x^\prime) \delta (t - t^\prime)
\end{align}

The fluctuation-dissipation theorem will ultimately show us that the noise strength $\lambda$ and the transport coefficient $\Gamma$ are related when the system is close to an equilibrium state by $\lambda = 2\Gamma$. To see how this might come about we start by evaluating the equilibrium pair correlation function using the partition functional of our theory

\subsection{The partition function route}

In equilibrium the probability of particular field configuration is given by the Boltzmann distribution.

\begin{equation}
\mathcal{P}_{eq}[\phi] = \f{e^{-\beta\F[\phi]}}{\mathcal{Z}[h(x)]}
\end{equation}

Where, $\mathcal{Z}[h(x)]$ is the partition functional and is given by a path integral over all field configurations.

\begin{equation}
\mathcal{Z}[h(x)] = \int \mathcal{D}[\phi] e^{-\beta\F[\phi]}
\end{equation}

Evaluation of the partition function is of some importance because it plays the role of a moment generating function.

\begin{equation}\label{gen}
\f{1}{\Z[h]}\f{\delta^n \Z[h]}{\delta h(x_1)...\delta h(x_n)} = \langle \phi(x_1)...\phi(x_n)\rangle
\end{equation}

In general the partition function cannot be computed directly, but in the special case of Gaussian free energies it can. To that end we consider expanding phi around an equilibrium solution, $\phi(x) = \phi_0 + \Delta\phi(x)$, and keeping terms to quadratic order in the free energy.

\begin{equation}
\beta\F[\Delta\phi] = \int dr \,\left\lbrace \f{1}{2}\Delta\phi(x) \left(r - \nabla^2 + \f{u}{2}\phi_0^2\right) \Delta\phi(x) - h(x)\Delta\phi(x) \right\rbrace
\end{equation}

Here the partition function is written in a suggestive form. As stated previously, functional integrals are difficult to compute in general, but Gaussian functional integrals do have a solution.

\subsubsection{Gaussian Functional Integrals}

Consider a functional integral of the following form.

\begin{equation}
\Z[h(x)] = \int \D[\phi] \exp\left\lbrace - \int dx \int dx^\prime \left[ \f{1}{2}\phi(x) \mathbf{K}(x, x^\prime) \phi(x^\prime)\right] +  \int dx \left[h(x) \phi(x)\right]\right\rbrace
\end{equation}

This integral is simply the continuum limit of a multivariable Gaussian integral,

\begin{equation}
\Z[\mathbf{h}] = \int \prod_i dx_i \exp \left\lbrace - \f{1}{2}\sum_i \sum_j x_i\, \mathbf{K}_{ij}\, x_j  + \sum_i h_i x_i\right\rbrace,
\end{equation}
For which the solution is,

\begin{equation}
\Z[\mathbf{h}] = \sqrt{\f{2\pi}{\det(\mathbf{K})}} \exp\left\lbrace \f{1}{2} \sum_i \sum_j h_i \mathbf{K}_{ij}^{-1} h_j\right\rbrace.
\end{equation}
In the continuum limit, the solution has an analogous form.

\begin{equation}\label{part}
\Z[h(x)] \propto \exp\left\lbrace \int dx \int dx^\prime \left[ \f{1}{2}h(x) \mathbf{K}^{-1}(x, x^\prime) h(x^\prime)\right] \right\rbrace
\end{equation}
Where $\mathbf{K}^{-1}$ is defined by,

\begin{equation}
\int dx^\prime \mathbf{K}(x, x^\prime)\mathbf{K}^{-1}(x^\prime, x^{\prime\prime}) = \delta(x - x^{\prime\prime}).
\end{equation}
Ultimately, we don't need to worry about the constant of proportionality in equation \ref{part} because we'll be dividing this contribution when calculating correlation functions.

\subsubsection{Computing the Pair correlation function in the Gaussian approximation}

To compute the pair correlation function we use the Fourier space variant of the partition function,

\begin{equation}
\Z[\tilde{h}(k)] \propto \exp\left\lbrace \f{1}{2}\int dk\,\f{h(k)h^{*}(k)}{r + \f{u}{2}\phi_0^2 +  \vert k \vert^2}\right\rbrace.
\end{equation}
The pair correlation function, $\langle \Delta\tilde{\phi}(k)\Delta\tilde{\phi}^{*}(k)\rangle$, is then computed using equation \ref{gen}.

\begin{equation}
\l\langle \Delta\fphi(k)\Delta\fphi^{*}(k^\prime) \r\rangle = \f{2\pi \delta(k+k^\prime)}{r + \f{u}{2}\phi_0^2 + \vert k \vert^2}
\end{equation}

\subsection{The Equation of Motion Route}

The equation of motion supplies a second method for evaluating the pair correlation function in equilibrium.

\begin{equation}
\f{\partial \phi}{\partial t} = -\Gamma\left((r-\nabla^2)\phi(x,t) + \f{u}{3!}\phi^3(x,t)\right) + \xi(x, t),
\end{equation}

Our equation of motion, can be linearized around an equilibrium solution, $\phi_0$, just as we did in the partition function route to the pair correlation function. In a similar vain, we will Fourier transform the equation of motion as well.

\begin{equation}
\f{\partial \Delta\fphi(k, t)}{\partial t} = -\Gamma\left((r + \f{u}{2}\phi_0 + \vert k \vert^2)\Delta\fphi(k,t)\right) + \xi(x,t)
\end{equation}
This equation can be solved formally using a Green's function solution.

\begin{equation}
\Delta\fphi(k, t) = e^{-\Omega t}\Delta\fphi(k, 0) + e^{-\Omega t}\int_0^t d\tau \,e^{\Omega \tau} \fxi(k, \tau)
\end{equation}
Where, $\Omega = \Gamma (r + \f{u}{2}\phi_0^2 + \vert k \vert^2)$.

\subsubsection{Computing the Pair correlation function}

We now have the tools to compute the dynamical pair correlation function, $\langle \Delta\fphi(k, t)\Delta\fphi(k^\prime, t^\prime) \rangle $, and in limit as time goes to infinity, we will have another expression for the equilibrium pair correlation function. We begin by inserting the Green's function solutions into the pair correlation expression

\begin{align}
\l\langle \Delta\fphi(k, t) \Delta\fphi(k^\prime, t^\prime)\r\rangle &=  e^{-\Omega(t+t^\prime)}\Delta\fphi(k, 0)\Delta\fphi(k^\prime, 0) \nonumber \\
 &+ e^{-\Omega (t+t^\prime)} \int_0^t d\tau \int_0^{t^\prime} d\tau^\prime e^{\Omega(\tau+\tau^\prime)}\l\langle \fxi(k, \tau) \fxi(k^\prime, \tau^\prime) \r\rangle.
\end{align}
Using the noise correlation we can compute the second term to find the final form of the dynamic correlation function.

\begin{align}
	\left\langle \Delta\fphi(k, t) \Delta\fphi(k^\prime, t^\prime)\right\rangle &=  e^{-\Omega(t+t^\prime)}\left(\Delta\fphi(k, 0)\Delta\fphi(k^\prime, 0) - \f{2\pi\delta(k+k^\prime)\lambda}{2\Omega}\right) \nonumber \\
	&+ \f{2\pi\delta(k+k^\prime)\lambda}{2\Omega}e^{-\Omega \vert t-t^\prime \vert}
\end{align}

Setting, $t = t^\prime$ and taking the limit as $t\rightarrow\infty$ we recover another form for the equilibrium pair correlation function.

\begin{equation}
	\left\langle \Delta\fphi(k, t) \Delta\fphi(k^\prime, t^\prime)\right\rangle = \f{2\pi\delta(k+k^\prime)\lambda}{2\Gamma(r + \f{u}{2}\phi_0^2 + \vert k\vert^2)}
\end{equation}

\subsection{Remarks}

Comparing with the result we got from the partition function route and the equation of motion route we see that for our answers to be equal we must have $\lambda = 2\Gamma$. It should be noted that this answer may seem to differ from other definitions of the fluctuation-dissipation theorem which state that $\lambda = 2k_b\,T\Gamma$. The discrepancy comes from what we mean by the coefficent $\Gamma$ and how we write the equation of motion. If we write the equation of motion as in equation \ref{eom}, the coefficient $\Gamma$ is the traditional Onsager transport coefficient and we recover traditional fluctuation-dissipation theorem.

\begin{equation}\label{eom}
	\f{\partial \phi(x,t)}{\partial t} = -\Gamma \left(\f{\delta \F[\phi]}{\delta \phi(x)}\right) - \xi(x,t)
\end{equation}

Comparing with our result we see that the factor of $k_bT$ is absorbed into our definition of the transport coefficient $\Gamma$.

\section{Fluctuation Dissipation Theorem in Dynamic Density Functional Theory}

In dynamic density functional theory (DDFT) we have an equation of motion of the following form,

\begin{equation}
	\f{\partial \rho(r, t)}{\partial t} = D_0 \nabla \cdot \l[\rho(r,t)\nabla \l(\f{\d \F[\rho]}{\d \rho}\r)\r] + \xi(r, t)
\end{equation}

Where, $D_0$ is the equilibrium diffusion constant and $\xi$ is the stochastic driving force. We assume once again that the driving force has no bias, but we now allow the noise strength to be a generic linear operator $\mathcal{L}$.

\begin{align}
	\langle \xi(r,t) \rangle &= 0 \\
	\l\langle \xi(r, t) \xi(r^\prime, t^\prime) \r\rangle &= \mathcal{L} \d (r-r^\prime) \d (t -t^\prime)
\end{align}

We ask ourselves now, what constrains can we apply to this operator if our system must decay to a Boltzmann distribution in equilibrium?

\subsection{Pair Correlation from the Partition Functional}

Just like with the $phi^4$ model we want to expand our free energy functional around an equilibrium solution. In this case our free energy functional is generic so this expansion is purely formal.

\begin{equation}
	\F[\rho] = \F_{eq} + \beta\int dr \l(\l.\f{\d \F[\rho]}{\d \rho(r)}\r)\r\vert_{\rho_{eq}}\Delta\rho(r) + \f{1}{2} \int dr \int dr^\prime \Delta\rho(r) \l(\l.\f{\d^2 \F[\rho]}{\d \rho(r) \d \rho(r^\prime)}\r)\r\vert_{\rho_{eq}} \Delta\rho(r^\prime)
\end{equation}

The first term we can neglect as it adds an overall scale to the partition function that will not affect any of moments. Second moment only shifts the average so we can ignore it as well and so we're left with a simple quadratic free energy once again.

\begin{equation}
	\F[\rho] = \f{1}{2}\int dr \int dr^\prime \Delta \rho(r) H(r, r^\prime) \Delta \rho(r^\prime)
\end{equation}

Where, $H(r, r^\prime)$ is the second functional derivative of the free energy functional in equilibrium. Computing the pair correlation function from the partition function yields, as might be expected,

\begin{equation}
	\l\langle \Delta\rho(r) \Delta\rho(r^\prime) \r\rangle = H(r, r^\prime)
\end{equation}

}

\ETDAppendix{Gaussian Functional Integrals}{
    \subsubsection{Gaussian Functional Integrals}

In the study of the statistical physics of fields we often encounter functional
integrals of the form,
%
\begin{equation}
    \Z[h(x)] = \int \D[\phi] \exp\left\lbrace - \int dx \int dx^\prime \left[
        \f{1}{2}\phi(x) \mathbf{K}(x, x^\prime) \phi(x^\prime)
    \right] + \int dx \left[h(x) \phi(x)\right]\right\rbrace.  
\end{equation}
%
Solutions to this integral are not only important in there own right but are
also the basis perturbative techniques. The detail of how to solve this
integral can be found in \cite{Kardar} and are repeated here for the
convenience of the reader.

This integral is simply the continuum limit of a multivariable Gaussian
integral,
%
\begin{equation}
    \Z[\mathbf{h}] = \int \prod_i dx_i \exp \left\lbrace 
        - \f{1}{2}\sum_i \sum_j x_i\, \mathbf{K}_{ij}\, x_j  
        + \sum_i h_i x_i\right\rbrace,
\end{equation}
%
For which the solution is,
%
\begin{equation}
    \Z[\mathbf{h}] = \sqrt{\f{2\pi}{\det(\mathbf{K})}} 
        \exp\left\lbrace \f{1}{2} \sum_i \sum_j 
        h_i \mathbf{K}_{ij}^{-1} h_j\right\rbrace.
\end{equation}
%
In the continuum limit, the solution has an analogous form.
%
\begin{equation}\label{part}
    \Z[h(x)] \propto \exp\left\lbrace \int dx \int dx^\prime 
        \left[ \f{1}{2}h(x) \mathbf{K}^{-1}(x, x^\prime) h(x^\prime)\right]
        \right\rbrace
\end{equation}
%
Where $\mathbf{K}^{-1}$ is defined by,
%
\begin{equation}
    \int dx^\prime \mathbf{K}(x, x^\prime)\mathbf{K}^{-1}
        (x^\prime, x^{\prime\prime}) = \delta(x - x^{\prime\prime}).
\end{equation}
%
Ultimately, we don't need to worry about the constant of proportionality in
equation \ref{part} because we'll be dividing this contribution when
calculating correlation functions.


}

\ETDAppendix{Binary Correlation Functions}{
    When developing the binary PFC model we often change variables from $\A$ and
$\B$ to $n$ and $c$.  This change of variable is helpful in identifying the
results of the PFC theory with established results in the field as
concentration and total density are more commonly used in the field of material
science. Computing the bulk terms (ie., $\Delta\F_{mix}[n, c]$ and
$\Delta\F_{id}[n]$ from equation \ref{binary_mixing} and \ref{binary_ideal}) is
a matter of substitution and simplification but computing the change of variables for excess free
energy can be more subtle. When computing the pair correlation
terms, careful application of our assumption that $c$ varies over a much longer
length scale than $n$ must be applied to get the correct solution. The goal,
ultimately, is to find $C_{n n}$, $C_{n c}$, $C_{c n}$ and $C_{c c}$ in the
following expression, 

[{\color{ForestGreen} Stopped reviewing here! continue hhere in future}]

%
\begin{gather}\label{1}
      \Delta \A \,\rho_0 C_{AA} \ast \Delta \A 
    + \Delta \A \,\rho_0 C_{AB} \ast \Delta \B 
    + \Delta \B \,\rho_0 C_{BA} \ast \Delta \A 
    + \Delta \B \,\rho_0 C_{BB} \ast \Delta \B = \\ \nonumber
       \l(n \,C_{nn} \ast n 
        + n \,C_{nc} \ast \Delta c 
        + \Delta c \,C_{cn}\ast n 
        + \Delta c \,C_{cc} \ast \Delta c\r).
\end{gather}
%
We begin by rewriting $\Delta \B$,
%
\begin{align*}
  \Delta \B &= \rho c - \rho_0 c_0 \\
        &= \rho c - \rho c_0 + \rho c_0 - \rho_0 c_0 \\
        &= \Delta \rho c + \rho_0 \Delta c,
\end{align*}
%
Followed by rewriting $\Delta \A$,
%
\begin{align*}
  \Delta \A &= \rho (1 - c) - \rho_0 (1 - c_0) \\
        &= \Delta \rho (1 - c) - \rho_0 \Delta c.
\end{align*}
%
With those forms established, we can expand $\Delta \B \,C_{BB} \ast \Delta \B$:
%
\begin{align}\label{this}
  \Delta \B C_{BB} \ast \Delta \B &= \l(\Delta \rho c + \rho_0 \Delta c \r) C_{BB} \ast \l(\Delta \rho c + \rho_0 \Delta c\r) \nonumber\\
                          &= \Delta \rho c \,C_{BB} \ast \l( \Delta \rho c\r) \nonumber\\
                          &+ \rho_0 \Delta c \,C_{BB} \ast \l( \Delta \rho c \r) \\
                          &+ \rho_0 \l(\Delta \rho c\r) \,C_{BB} \ast \Delta c \nonumber\\
                          &+ \rho_0^2 \Delta c \,C_{BB} \ast \Delta c. \nonumber
\end{align}
%
If we examine one term in this expansion in detail, we note that we can
simplify by using the long wavelength approximation for the concentration
field,
%
\begin{align}
  \Delta \rho c \, C_{BB} \ast \Delta \rho c &= \Delta \rho(r) c(r) \int dr^\prime C_{BB}(r - r^\prime) \Delta \rho(r^\prime) c(r^\prime) \nonumber \\
                                     &\approx \Delta \rho(r) c^2(r) \int dr^\prime C_{BB}(r - r^\prime) \Delta \rho(r^\prime).
\end{align}
%
This is because the concentration field can be considered ostensibly constant
over the length scale in which $C_{BB}(r)$ varies. Recall that the pair
correlation function typically decays to zero on the order of several particle
radii. Using this approximation we can rewrite equation \ref{this} as,
%
\begin{align}
  \Delta \B \, C_{BB} \ast \Delta \B &= \Delta \rho \l(c^2 \,C_{BB}\r) \ast \Delta \rho \nonumber\\
                             &+ \rho_0 \Delta c \l(c \,C_{BB}\r) \ast \Delta \rho c \\
                             &+ \rho_0 \Delta \rho \l(c \,C_{BB}\r) \ast \Delta c \nonumber\\
                             &+ \rho_0^2 \Delta c \,C_{BB} \ast \Delta c. \nonumber
\end{align}
%
Repeating this procedure with the remaining three terms and then regrouping we
can easily identify the required pair correlations.\footnote{Note that we may
also take advantage of the fact that $C_{AB} = C_{BA}$.}
%
\begin{gather}
  C_{nn} = \rho_0 \l( c^2 \, C_{BB} + (1 - c)^2 \, C_{AA} + 2c(1-c)\,C_{AB}\r) \\
  C_{nc} = C_{cn} = \rho_0 \l( c\,C_{BB} - (1-c)\,C_{AA} + (1 - 2c) \, C_{AB} \r) \\
  C_{cc} = \rho_0 \l( C_{BB} + C_{AA} - 2 C_{AB} \r)
\end{gather}


}

\bibHeading{References}
\bibliography{references}
\bibliographystyle{plain}

\index[abbr]{PFC: Phase Field Crystal}
\index[abbr]{CDFT: Classical Density Functional Theory}

\printindex[keylist]{Index}{Index}{}
\printindex[abbr]{KEY TO ABBREVIATIONS}{KEY TO ABBREVIATIONS}{}

\end{document}
